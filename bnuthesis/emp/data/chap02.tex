% !TeX root = ../bnuthesis-example.tex

\chapter{DHW形式}
早期常用的研究真空对生成问题的方法包括有效作用量、固有时间方法和Wentzel-Kramers-Brillouin (WKB) 近似。已经开发出许多更简单、更实用的半经典近似和量子动力学方法,包括世界线瞬子技术 (worldline instantons,WI) ,S矩阵(S-matrix),量子Vlasov方程 (QVE) 和 Wigner函数形式\cite{2017xie}。DHW 形式不仅可以用来研究在存在空间非均匀含时电场的情况下的真空对生成问题,也可以用来研究在存在磁场的情况下的对生成过程。事实上,DHW形式可以理论上研究在任何类型的电磁场下的真空对生成问题。然而,对于高维场,它通常受到计算机存储大小和计算能力的限制。值得注意的是,对于空间均匀含时电场,DHW 形式可以简化为QVE。在众多关于在电子对真空中生成正负电子对的方法中,本论文选择了DHW形式。这种方法起始于狄拉克密度算符,引入了维格纳函数,最终导出了完整的DHW运动方程。由于关于DHW形式的推导众多且复杂,本文在这里仅仅对空间均匀含时电场下的DHW形式做简单介绍。

\section{Winger算符}

量子力学的相空间表述最初是为了研究经典统计力学的量子修正而引入的,其中,维格纳函数是为此目的引入的最著名的相空间表述之一。相空间表述的优点不仅在于它提供了一种用经典语言描述量子系统的方式,而且还揭示了粒子运动的实时动态。如今,维格纳函数广泛用于处理非相对论量子物理问题。

维格纳函数的形式是建立在厄米QED拉格朗日量的基础上的:
\begin{equation}
    \mathcal{\hat{L}}(\hat{\Psi},\bar{\hat{\Psi } },\hat{\overrightarrow{A} } ) = \frac12\left[\bar{\hat{\Psi }}\gamma^\mu [i \partial_\mu-eA_\mu]\hat{\Psi}-\left([i\partial_\mu+eA_\mu]\bar{\hat{\Psi }}\right)\gamma^\mu\hat{\Psi}\right]-m\bar{\hat{\Psi }}\hat{\Psi}-\frac14F^{\mu\nu}F_{\mu\nu},
\end{equation}
由此得到旋量的狄拉克方程$\hat{\Psi}$及其伴随旋量的伴随形式 $\bar{\hat{\Psi}}=\hat{\Psi}^\dagger\gamma^0$,并且
\begin{equation}
\begin{aligned}
&\partial_{t} \hat{\Psi}(\vec{x}, t)=-\gamma^{0} \vec{\gamma}\left[\vec{\nabla}_{\vec{x}}-i e \hat{\vec{A}}(\vec{x}, t)\right] \hat{\Psi}(\vec{x}, t)-i m \gamma^{0} \hat{\Psi}(\vec{x}, t) \\
&\partial_{t} \overline{\hat{\Psi}}(\vec{x}, t)=-\left[\nabla_{\vec{x}} \hat{\hat{\Psi}}(\vec{x}, t)+i e \hat{\hat{\Psi}}(\vec{x}, t) \hat{\vec{A}}(\vec{x}, t)\right] \cdot \vec{\gamma} \gamma^{0}+i m \hat{\hat{\Psi}}(\vec{x}, t) \gamma^{0}
\end{aligned}        
\end{equation}
注意,这里使用了爱因斯坦的求和约定。电磁场将被视为一个经典的外场,它的运动方程将被简化为$\vec{E}$和$\vec{B}$服从麦克斯韦方程的要求:
\begin{equation}
\operatorname{div} \vec{E} = \rho \quad \operatorname{rot} \vec{E} = -\dot{\vec{B}} \quad
\operatorname{div} \vec{B} = 0 \quad \operatorname{rot} \vec{B} = \dot{\vec{E}}+\mu_{0} j .
\end{equation}
从费米子场算子$\hat{\Psi}$及其共轭算子$\hat{\Psi}^\dagger$出发,
\begin{equation}
C_{ab}^{\pm}(t,\vec{x_{1}},\vec{x_{2}}):=\langle0|\hat{\Psi}_{a}(t,\vec{x_{1}})\hat{\Psi}_{b}^{\dagger}(t,\vec{x_{2}})|0\rangle\pm\langle0|\hat{\Psi}_{b}^{\dagger}(t,\vec{x_{2}})\hat{\Psi}_{a}(t,\vec{x_{1}})|0\rangle,
\end{equation}
可以用来建立维格纳函数。由于$C^+$是域算子的反对易子的期望值,根据定义它是$\delta_{ab}\cdot\allowbreak\delta(\vec{x_2}-\vec{x_1})$函数,所以只有$C^-$包含非平凡信息。因此,场算符的换向子将是维格纳函数定义中的起点。引入实验室坐标系质心坐标$\vec{x}$,相对坐标$\vec{s}$ 
\begin{equation}
     \vec{x} : =  \frac12\left( \vec{x_1} + \vec{x_2} \right) \\
  \vec{s} : = \vec{x_2} - \vec{x_1}\,.
\end{equation}
现在,Wigner算子$\hat{\mathcal{W}}$被定义为Heisenberg图中两个Dirac场算子的等时间密度算子的傅里叶(Wigner)变换。由于我们对真空的对产生感兴趣,Wigner函数是通过取Wigner算子的真空期望值来定义的$\langle0|\diamond|0\rangle $。注意使用Dirac共轭场算子$\bar{\hat{\Psi} }=\hat{\Psi}^\dagger\gamma^0$代替伴随场算子,使得Wigner算子在洛伦兹变换下齐次变换

\begin{equation}\label{2-6}
 \begin{aligned}
&\mathcal{W}(t,\vec{x},\vec{p}) :=\langle0|\hat{\mathcal{W}}(t,\vec{x},\vec{p})|0\rangle   \\
& \hat{\mathcal{W}}_{ab}(t,\vec{x},\vec{p}):=-\frac12\int d\overrightarrow{s}\mathrm{~e}^{-\frac{\mathrm{i}}\hbar\vec{p}\cdot\vec{s}}\hat{\mathcal{C}}_{ab}(t,\overrightarrow{x},\overrightarrow{s})  \\
&\hat{\mathcal{C}}_{ab}(t,\vec{x},\vec{s}) :=\hat{\Phi}(t,\vec{x},\vec{s})\left[\hat{\Psi}_{a}(t,\vec{x}+\vec{s}/2),\overline{\hat{\Psi}_{b}}(t,\vec{x}-\vec{s}/2)\right]  \\
&\hat{\Phi}(t,\vec{x},\vec{s}) :=\mathcal{P}\text{e}^{-\mathrm{i}e\int_{\vec{x}+\vec{s}/2}^{\vec{x}-\vec{s}/2}\vec{\hat{A}}(t,\vec{x}^{\prime})\cdot d\vec{x}^{\prime}} .
\end{aligned}
\end{equation}

引入Wilson线因子$\hat{\Phi}(t,\vec{x},\vec{s})$来实现规范不变性。
对电磁场进行经典处理后,路径排序将被去掉。选择一条直线作为积分路径,可以保证对$\vec{p}$作为动能的正确解释。请注意,这不是依赖于时空点和四动量的协变维格纳函数的定义,而是取能量平均值的等时维格纳函数。傅里叶变换可以理解为测量两点相关相对于原点$\vec{x}$的平面波含量。然后将平面波的振幅解释为在位置 $\vec{x}$处动量为$\vec{p}$的粒子的准概率密度。因此Wigner函数可用于计算期望值任何可观察到的,用场算符表示的。如果将变量$\vec{x}$或$\vec{p}$ 中的一个积分出来,则剩余的函数为正,可以解释为粒子密度。

\section{运动方程}

在定义方程式\ref{2-6} 中,对每个场算符使用海森堡运动方程,可以推导出 Wigner函数的运动方程。结果是一个耦合方程组,它不仅描述 Wigner函数的演化,还描述了电磁场的演化。这个被称为 BBGKY方程组,以 N. Bogoliubov、M. Born、H.Green、G. Kirkwood 和 J.Yvon 的名字命名。为了进行实际的数值计算,这个方程组可以被截断。这种Hartree 型或平均场近似,将在下面解释,实际上是到目前为止在 Wigner 方法中使用的唯一近似。

我们用期望值的乘积来代替算子的期望值
\begin{equation}
 \begin{aligned}
&\langle 0|\hat{E} \hat{\mathcal{O}}| 0\rangle \rightarrow\langle 0|\hat{E}| 0\rangle \cdot\langle 0|\hat{\mathcal{O}}| 0\rangle, \\
&\langle 0|\hat{B} \hat{\mathcal{O}}| 0\rangle \rightarrow\langle 0|\hat{B}| 0\rangle \cdot\langle 0|\hat{\mathcal{O}}| 0\rangle .
\end{aligned}
\end{equation}

BBGKY层次结构的无穷方程组在Wigner函数的第一能级和电磁场的第零能级被截断,因为后者被视为经典背景场。然后可以将Wigner函数的动力学方程写成:
\begin{equation}\label{2-8}
    D_t\mathcal{W}=-\frac12\vec{D}_{\vec{x}}\left[\gamma^0\vec{\gamma},\mathcal{W}\right]-\mathrm{i}m\left[\gamma^0,\mathcal{W}\right]-\mathrm{i}\vec{P}\left\{\gamma^0\vec{\gamma},\mathcal{W}\right\}
\end{equation}
其中伪微分算子
\begin{equation}
 \begin{aligned}
&D_{t} =\partial_{t}+e\int\limits_{-1/2}d\lambda\vec{E}(t,\vec{x}+\mathrm{i}\lambda\vec{\nabla}_{\vec{p}})\cdot\vec{\nabla}_{\vec{p}}, \\
&\vec{D}_{\vec{x}} =\vec{\nabla}_{\vec{x}}+e\intop_{-1/2}^{1/2}d\lambda\vec{B}(t,\vec{x}+\mathrm{i}\lambda\vec{\nabla}_{\vec{p}})\times\vec{\nabla}_{\vec{p}}, \\
&\vec{P} i=\vec{p}-\mathrm{i}e\int\limits_{-1/2}^{1/2}d\lambda\lambda\vec{B}(t,\vec{x}+\mathrm{i}\lambda\vec{\nabla}_{\vec{p}})\times\vec{\nabla}_{\vec{p}}. 
\end{aligned}
\end{equation}
在这里,我们使用约定$\left \{  {\gamma^\mu}{\gamma^\nu} \right \}=+2\eta^{\mu\nu}=+2\operatorname{diag}(1,-1,-1,-1)$和使用时间规范$A_0=0$。电场$\vec{E}$和磁场$\vec{B}$由下式给出:
\begin{equation}
    \begin{aligned}\vec{E}&=-\partial_t\vec{A}\\\vec{B}&=\vec{\nabla}_{\vec{x}}\times\vec{A}.\end{aligned}
\end{equation}
在费曼图的语言中,平均场近似对应于忽略辐射修正,这是由精细结构常数 
$\alpha$很小所导致的。Wigner函数可以用Dirac双线性函数的完备基进行分解$(\mathbb{1},\gamma^5,\gamma^\mu,\gamma^\mu\gamma^5,\sigma^{\mu\nu}:=\frac{\mathrm{i}}{2}\left[\gamma^\mu,\gamma^\nu\right])$,
\begin{equation}\label{2-11}
\mathcal{W}=\frac14\left(\mathbb{1}\text{s}+\text{i}\gamma_5\text{p}+\gamma^\mu\text{v}_\mu+\gamma^\mu\gamma_5\text{al}_\mu+\sigma^{\mu\nu}\text{t}_{\mu\nu}\right)
\end{equation}
与相应的变换系数函数$\mathbb{s},\mathbb{p},\mathbb{v_\mu},\mathbb{a_\mu},\mathbb{t_{\mu\nu}}$。将方程\ref{2-11}代入方程\ref{2-8},后者可以分解,得到
\begin{equation}\label{2-12}
\begin{aligned}
&D_{t} s=\quad+2 \vec{P} \cdot \overrightarrow{\mathbb{t}}^{1} \\
&D_{t} \mathbb{P}=\quad-2 \vec{P} \cdot \overrightarrow{\mathbb{t}}^{2}-2 m \mathrm{a}^{0} \\
&D_{t} \mathbb{\mathbb { v }}^{0}=-\vec{D}_{\vec{x}} \overrightarrow{\mathbb{v}} \\
&D_{t} \mathrm{a}^{0}=-\vec{D}_{\vec{x}} \vec{a} \quad+2 m \mathbb{p} \\
&D_{t} \overrightarrow{\mathbb{V}}=-\vec{D}_{\vec{x}} \mathbb{v}^{0} \quad-2 \vec{P} \times \vec{a}-2 m \overrightarrow{\mathbb{t}}^{1} \\
&D_{t} \overrightarrow{\mathrm{a}}=-\vec{D}_{\vec{x}} \mathrm{a}^{0} \quad-2 \vec{P} \times \overrightarrow{\mathrm{v}} \\
&D_{t} \overrightarrow{\mathbb{t}}^{1}=-\vec{D}_{\vec{x}} \times \overrightarrow{\mathbb{t}}^{2}-2 \vec{P}\mathbb{s} \quad+2 m \overrightarrow{\mathbb{v}} \\
&D_{t} \overrightarrow{\mathbb{t}}^{2}=+\vec{D}_{\vec{x}} \times \overrightarrow{\mathbb{t}}^{1}+2 \vec{P} \mathbb{p} \\
\end{aligned}
\end{equation}
两个向量$\vec{\mathbb{t}}^{1/2}$包含了反对称张量的分量 ${\mathbb{t}}_{\mu\nu}$

\begin{equation}
\vec {\mathbb{t}}^{1} : = 2 \mathbb{t}^{i0}\vec{e_i},  
\vec {\mathbb{t}}^{2} : = \mathbb{t}_{ij}\epsilon^{ijk}\vec{e_k}\,.
\end{equation}
当时间$t\to -\infty$,$\vec{E}$ and $\vec{B}$ 逐渐消失,初始条件由具有非零分量的真空解给出

\begin{equation}\label{2-14}
\mathbb{s}_{\mathrm{vac.}}=\frac{-2m}{\omega(\vec{p})},
\mathbb{v}_{\mathrm{vac.}}=\frac{-2\vec{p}}{\omega(\vec{p})},
\end{equation}
其中$\omega^2: = \vec{p}^2+m^2$.
真空解也可以用矩阵形式写成:
\begin{equation}
\mathcal{W}_{\mathrm{vac}}=\frac14(\mathbb{1}\mathbb{s}_{\mathrm{vac}}+\gamma^\mu\mathbb{V}_{\mathrm{vac}\mu})=-\frac m{2\omega}\mathbb{1}+\frac{\vec{\gamma}\cdot\vec{p}}{2\omega}.
\end{equation}
运动方程式\ref{2-12}结合初始条件式\ref{2-14}定义初值问题。

\section{可观测量} 
利用Wigner函数的定义和QED的对称性,我们导出守恒量

电荷:
\begin{equation}
\mathcal{Q}=e\int\mathrm{d}\Gamma\mathbb{v}_0(t,\vec{x},\vec{p})   
\end{equation}
能量:
\begin{equation}\label{2-17}
\begin{aligned}\mathcal{E}&=\int\mathrm{d}\Gamma\left(\underbrace{\vec{p}\cdot\vec{v}(t,\vec{x},\vec{p}^{\prime})+m\operatorname{s}(t,\vec{x},\vec{p}^{\prime})}_{\epsilon(t,\vec{x},\vec{p})}\right)\\&+\frac12\int\mathrm{d}^3x\left(\left|\vec{E}(t,\vec{x}^{\prime})\right|^2+\left|\vec{B}(t,\vec{x}^{\prime})\right|^2\right)\end{aligned}
\end{equation}
动量:
\begin{equation}  \vec{P}=\int\mathrm{d}\Gamma\vec{p}\mathbb{v}_0(t,\vec{x},\vec{p})+\int\mathrm{d}^3x\vec{E}(t,\vec{x})\times\vec{B}(t,\vec{x}),
\end{equation}
角动量:
\begin{equation}\begin{aligned}\vec{S}&=\int\mathrm{d}\Gamma\left(\vec{x}\times\vec{p}\mathbb{v}_0(t,\vec{x},\vec{p})-\frac12\vec{\mathbb{a}} ( t , \vec { x },\vec{p})\right)\\&+\int\mathrm{d}^3x\vec{x}\times\left(\vec{E}(t,\vec{x})\times\vec{B}(t,\vec{x})\right)\end{aligned}\end{equation}
洛伦兹boost因子:
\begin{equation}\begin{aligned}\vec{K}=t\vec{P}-\int\mathrm{d}\Gamma\left.\vec{x}\left(\vec{p}\cdot\vec{v}(t,\vec{x},\vec{p})+m\operatorname{s}(t,\vec{x},\vec{p})\right)\right.\\-\frac12\int\mathrm{d}^3x\left(\left|\vec{E}(t,\vec{x})\right|^2+\left|\vec{B}(t,\vec{x})\right|^2\right)\end{aligned}.
\end{equation}
相空间测量值$d\Gamma$由$\mathrm{d}\Gamma=\mathrm{d}^3\vec{x}\frac{\mathrm{d}^3\vec{p}}{\left(2\pi\right)^3}$得到。
方程\ref{2-17}中被积函数中的$\epsilon(t,\vec{x},\vec{p})$是特别有趣的,因为它可以解释为费米子场的(相空间)能量密度
\begin{equation}\label{2-21}
     \epsilon(t,\vec{x},\vec{p}) = \vec{p}\cdot \vec{\mathbb{v}}(t,\vec{x},\vec{p})+m\,\mathbb{s}(t,\vec{x},\vec{p}) .
\end{equation}
由能量密度减去真空解$\epsilon_\mathrm{vac} =  m \mathbb{s}_\mathrm{vac} +\vec{p}\cdot\vec{\mathbb{v}}_\mathrm{vac} = -2\omega $并且将粒子对的能量结果归一化,这样我们就可以得到粒子分布函数
\begin{equation}\label{2-22}
f:=\frac1{2\omega}\left(\epsilon-\epsilon_{\mathrm{vac}}\right)=\frac1{2\omega}\epsilon+1.
\end{equation}
在空间均匀单方向的电场中,这个分布函数$f$与QVE中的分布函数的定义相同。注意,可以用更常见的表达\ref{2-21}:
\begin{equation}
\epsilon[\mathcal{W}]=\operatorname{tr}[\mathcal{W}(m\mathbb{1}+\vec{p}\cdot\vec{\gamma})].
\end{equation}
因此,式\ref{2-22}可以用维格纳函数的投影来表示
\begin{equation}
f[\mathcal{W}-\mathcal{W}_{\mathrm{vac.}}]=\frac1{2\omega}\operatorname{tr}\left[\left(\mathcal{W}-\mathcal{W}_{\mathrm{vac.}}\right)(m\mathbb{1}+\vec{p}\cdot\vec{\gamma})\right].
\end{equation}
必须强调的是,粒子解释只有在$\vec{E}=\vec{B}=0$时才有效。对于本文所讨论的情况,这在大的时间是正确的。例如,从真空中产生的粒子数(粒子数目=反粒子数目=对数目)表达式如下:
\begin{equation}
N=\lim_{t\to\infty}\int\mathrm{d}\Gamma f(t,\vec{x},\vec{p}).
\end{equation}
当假设空间均匀性时,分布函数不依赖于位置,并且在空间上积分没有意义,因为这将产生无穷大。然后对动量进行积分就得到对产生数密度:
\begin{equation}
\mathcal{N}:=\lim_{t\to\infty}\int\frac{\operatorname{d}^3\vec{p}}{\left(2\pi\right)^3}f(t,\vec{p}).
\end{equation}
在一些数值计算中,没有计算所有$\vec{p}$的渐近分布函数,而只计算$p_z=0$的切面。在这种情况下,$p_x$-$p_y$平面上的粒子产率称为$\mathcal{N}_{xy}$并定义为
\begin{equation}
\mathcal{N}_{xy}:=\lim_{t\to\infty}\int\frac{\mathrm{d}p_x}{2\pi}\int\frac{\mathrm{d}p_y}{2\pi}\left.f(t,\vec{p})\right|_{p_z=0}.
\end{equation}
