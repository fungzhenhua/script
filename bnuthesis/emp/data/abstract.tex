% !TeX root = ../bnuthesis-example.tex

% 中英文摘要和关键字

\begin{abstract}
  在量子电动力学中,强场下真空中正负电子对的产生是最引人注目的现象之一。随着激光技术的飞速发展,实验室中可产生的电磁场强度大幅提高,从而使得强外场下真空中产生正负电子对的问题再次成为研究的热点。研究者不断改进理论方法和改良物理模型,希望实现降低正负电子对产生阈值和提高对产生率的目的。对不同外场构型中正负电子对产生的研究有助于加深对电子对产生过程的理解,同时能够之后的实验观测提供对应的理论和研究思路。
  
  本论文主要研究脉冲形状和频率啁啾效应对正负电子对产生的影响。我们采用Wigner函数形式来研究真空中正负电子对的产生问题。Drac-Heisenberg-Wigner(DHW)形式被广泛应用于研究空间均匀含时电场下正负电子对的产生。具体研究内容如下:
  
  1. 非对称脉冲形状对正负电子对产生的影响:我们考察了不同参数场中正负电子对产生的情况,包括无啁啾、小频率啁啾和大频率不对称啁啾场,采用实时DHW形式。研究发现,随着下降脉冲长度的缩短,干涉效应逐渐消失,动量谱中的峰值集中在谱的左侧。随着下降脉冲长度的延长,动量谱中出现了不完整的多环结构。此外,我们发现粒子的数密度对脉冲的非对称性非常敏感。当使用特定的频率啁啾时,长下降脉冲可以将数密度显著提高四个数量级以上。这些结果突显了动力学辅助机制和频率啁啾效应对正负电子对产生的作用。
  
  2. 对称频率啁啾电场对正负电子对产生的影响:在脉冲形状相同时,我们进一步研究了对称频率啁啾电场对正负电子对产生的影响。对于对称频率的啁啾电场,随着啁啾参数的增加,动量谱中的峰逐渐增大和分裂,同时产生强烈的干涉效应。这一现象与不对称频率啁啾电场中的变化趋势一致,并且在长下降脉冲下更加明显。这些现象不仅体现了动力学辅助的Schwinger机制的影响,还进一步说明了频率啁啾效应对正负电子对产生具有巨大的影响。

  % 关键词用“英文逗号”分隔,输出时会自动处理为正确的分隔符
  \bnusetup{
    keywords = {强场物理, 真空正负电子对, Schwinger效应, 频率啁啾, DHW形式},
  }
\end{abstract}

\begin{abstract*}
Vacuum pair production of electron-positron pairs in strong fields is one of the most notable phenomena in quantum electrodynamics. With the rapid advancement of laser technology, the intensity of electromagnetic fields producible in laboratories has significantly increased, reinvigorating research into the production of electron-positron pairs in strong external fields. Researchers continuously refine theoretical approaches and enhance physical models, aiming to lower the threshold for electron-positron pair production and increase its rate. Studying the production of electron-positron pairs in different field configurations contributes to a deeper understanding of the pair production process and provides theoretical frameworks and research directions for subsequent experimental observations.

This study primarily investigates the impact of pulse shape and frequency chirp effects on the production of electron-positron pairs. We employ the Wigner function formalism to examine the generation of electron-positron pairs in vacuum. The Drac-Heisenberg-Wigner (DHW) formalism is extensively applied to investigate the production of electron-positron pairs under spatially homogeneous time-dependent electric fields. The specific research contents are outlined as follows:

1. Influence of asymmetric pulse shapes on electron-positron pair production: We examine the production of electron-positron pairs in different parameter regimes, including chirp-free, small frequency chirp, and large asymmetric frequency chirp fields, using real-time DHW formalism. Our investigation reveals that as the duration of the falling pulse shortens, interference effects gradually diminish, and the peak in the momentum spectrum concentrates on the left side of the spectrum. With prolonged falling pulse duration, incomplete multi-ring structures emerge in the momentum spectrum. Additionally, we observe that the asymmetry of the pulse significantly impacts the particle number density. When employing specific frequency chirps, long falling pulses can markedly increase the particle number density by more than four orders of magnitude. These findings underscore the role of dynamically assisted mechanism and frequency chirp effects on pair creation. 

2. Influence of symmetric frequency chirp fields on electron-positron pair production: Under identical pulse shapes, we further investigate the impact of symmetric frequency chirp fields on the production of electron-positron pairs. For symmetric frequency chirp fields, as the chirp parameter increases, the peaks in the momentum spectrum gradually enlarge and split, accompanied by strong interference effects. This phenomenon aligns with the trends observed in asymmetric frequency chirp fields and becomes more pronounced under long falling pulses. These observations not only reflect the influence of the dynamically assisted Schwinger mechanism but also underscore the significant impact of frequency chirp effects on electron-positron pair production.
  % Use comma as separator when inputting
  % 英文关键词首字母请大写
  \bnusetup{
    keywords* = {strong-field physics, vacuum pair production, Schwinger effect, frequency chirp, DHW formalism},
  }
\end{abstract*}
