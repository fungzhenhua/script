\markdownRendererHeadingOne{Changelog}\markdownRendererInterblockSeparator
{}此处记载了 ThuThesis 中所有值得留意的改动,格式参照 \markdownRendererLink{Keep a Changelog}{https://keepachangelog.com/en/1.0.0/}{https://keepachangelog.com/en/1.0.0/}{}。\markdownRendererInterblockSeparator
{}点击版本号即可在 GitHub 上查看相邻版本间的代码变动。\markdownRendererInterblockSeparator
{}\markdownRendererHeadingTwo{\markdownRendererLink{Unreleased}{https://github.com/tuna/thuthesis/compare/v7.4.0...HEAD}{https://github.com/tuna/thuthesis/compare/v7.4.0...HEAD}{}}\markdownRendererInterblockSeparator
{}\markdownRendererHeadingThree{Added}\markdownRendererInterblockSeparator
{}\markdownRendererUlBegin
\markdownRendererUlItem 本科生的附录(调研阅读报告和书面翻译)支持 \markdownRendererCodeSpan{biblatex}(\markdownRendererLink{\markdownRendererHash{}893}{https://github.com/tuna/thuthesis/issues/893}{https://github.com/tuna/thuthesis/issues/893}{})。\markdownRendererUlItemEnd 
\markdownRendererUlEnd \markdownRendererInterblockSeparator
{}\markdownRendererHeadingThree{Changed}\markdownRendererInterblockSeparator
{}\markdownRendererUlBegin
\markdownRendererUlItem 不再插入 PDF 版书脊,改为编译生成(\markdownRendererLink{\markdownRendererHash{}551}{https://github.com/tuna/thuthesis/issues/551}{https://github.com/tuna/thuthesis/issues/551}{})。\markdownRendererUlItemEnd 
\markdownRendererUlItem 本科生附录的参考文献编译方式改为 \markdownRendererCodeSpan{bibtex thuthesis-appendix-\markdownRendererLeftBrace{}a,b,c...\markdownRendererRightBrace{}},同研究生一致。\markdownRendererUlItemEnd 
\markdownRendererUlItem 封面的学科门类的字号改为三号(16bp),同步《指南》2023 年 3 月版的更改(\markdownRendererLink{\markdownRendererHash{}899}{https://github.com/tuna/thuthesis/issues/899}{https://github.com/tuna/thuthesis/issues/899}{})。\markdownRendererUlItemEnd 
\markdownRendererUlItem 统一并简化封面的布局(\markdownRendererLink{\markdownRendererHash{}900}{https://github.com/tuna/thuthesis/issues/900}{https://github.com/tuna/thuthesis/issues/900}{})。\markdownRendererUlItemEnd 
\markdownRendererUlItem 英文封面的导师姓名居中对齐(\markdownRendererLink{\markdownRendererHash{}883}{https://github.com/tuna/thuthesis/issues/883}{https://github.com/tuna/thuthesis/issues/883}{})。\markdownRendererUlItemEnd 
\markdownRendererUlEnd \markdownRendererInterblockSeparator
{}\markdownRendererHeadingThree{Fixed}\markdownRendererInterblockSeparator
{}\markdownRendererUlBegin
\markdownRendererUlItem 修正封面的职称字距问题(\markdownRendererLink{\markdownRendererHash{}879}{https://github.com/tuna/thuthesis/discussions/879}{https://github.com/tuna/thuthesis/discussions/879}{})。\markdownRendererUlItemEnd 
\markdownRendererUlItem 解决了本科生附录的 \markdownRendererCodeSpan{\markdownRendererBackslash{}printbibliography} 报错的问题(\markdownRendererLink{\markdownRendererHash{}882}{https://github.com/tuna/thuthesis/issues/882}{https://github.com/tuna/thuthesis/issues/882}{})。\markdownRendererUlItemEnd 
\markdownRendererUlEnd \markdownRendererInterblockSeparator
{}\markdownRendererHeadingTwo{\markdownRendererLink{v7.4.0}{https://github.com/tuna/thuthesis/compare/v7.3.2...v7.4.0}{https://github.com/tuna/thuthesis/compare/v7.3.2...v7.4.0}{} - 2023-05-15}\markdownRendererInterblockSeparator
{}\markdownRendererHeadingThree{Added}\markdownRendererInterblockSeparator
{}\markdownRendererUlBegin
\markdownRendererUlItem 允许本科生附录翻译的摘要中使用 \markdownRendererCodeSpan{\markdownRendererBackslash{}thusetup\markdownRendererLeftBrace{}keywords = *\markdownRendererRightBrace{}} 设置关键词(\markdownRendererLink{\markdownRendererHash{}865}{https://github.com/tuna/thuthesis/issues/865}{https://github.com/tuna/thuthesis/issues/865}{})。\markdownRendererUlItemEnd 
\markdownRendererUlItem 添加选项 \markdownRendererCodeSpan{degree-category} 和 \markdownRendererCodeSpan{degree-category*} 设置学科门类(\markdownRendererLink{\markdownRendererHash{}840}{https://github.com/tuna/thuthesis/issues/840}{https://github.com/tuna/thuthesis/issues/840}{})。\markdownRendererUlItemEnd 
\markdownRendererUlItem 添加选项 \markdownRendererCodeSpan{professional-field} 和 \markdownRendererCodeSpan{professional-field*} 设置专业领域(\markdownRendererLink{\markdownRendererHash{}840}{https://github.com/tuna/thuthesis/issues/840}{https://github.com/tuna/thuthesis/issues/840}{})。\markdownRendererUlItemEnd 
\markdownRendererUlItem 添加选项 \markdownRendererCodeSpan{engineering-field} 和 \markdownRendererCodeSpan{engineering-field*} 设置工业领域(\markdownRendererLink{\markdownRendererHash{}840}{https://github.com/tuna/thuthesis/issues/840}{https://github.com/tuna/thuthesis/issues/840}{})。\markdownRendererUlItemEnd 
\markdownRendererUlEnd \markdownRendererInterblockSeparator
{}\markdownRendererHeadingThree{Changed}\markdownRendererInterblockSeparator
{}\markdownRendererUlBegin
\markdownRendererUlItem 专业学位的“工程领域”改为“专业领域”,同步《指南》2023 年 3 月版的更改(\markdownRendererLink{\markdownRendererHash{}862}{https://github.com/tuna/thuthesis/issues/862}{https://github.com/tuna/thuthesis/issues/862}{})。\markdownRendererUlItemEnd 
\markdownRendererUlItem 附录中的参考文献另行编号(\markdownRendererLink{\markdownRendererHash{}837}{https://github.com/tuna/thuthesis/issues/837}{https://github.com/tuna/thuthesis/issues/837}{},感谢 \markdownRendererLink{@hushidong}{https://github.com/hushidong}{https://github.com/hushidong}{} 和 \markdownRendererLink{@atxy-blip}{https://github.com/atxy-blip}{https://github.com/atxy-blip}{})。\markdownRendererUlItemEnd 
\markdownRendererUlEnd \markdownRendererInterblockSeparator
{}\markdownRendererHeadingThree{Deprecated}\markdownRendererInterblockSeparator
{}\markdownRendererUlBegin
\markdownRendererUlItem 选项 \markdownRendererCodeSpan{degree-name} 和 \markdownRendererCodeSpan{degree-name*} 已经过时(\markdownRendererLink{\markdownRendererHash{}840}{https://github.com/tuna/thuthesis/issues/840}{https://github.com/tuna/thuthesis/issues/840}{})。\markdownRendererUlItemEnd 
\markdownRendererUlEnd \markdownRendererInterblockSeparator
{}\markdownRendererHeadingThree{Fixed}\markdownRendererInterblockSeparator
{}\markdownRendererUlBegin
\markdownRendererUlItem 修正研究生“学术成果”列表的行距(\markdownRendererLink{\markdownRendererHash{}850}{https://github.com/tuna/thuthesis/issues/850}{https://github.com/tuna/thuthesis/issues/850}{})。\markdownRendererUlItemEnd 
\markdownRendererUlItem 修正封面的布局,同步《指南》2023 年 3 月版的更改(\markdownRendererLink{\markdownRendererHash{}861}{https://github.com/tuna/thuthesis/issues/861}{https://github.com/tuna/thuthesis/issues/861}{})。\markdownRendererUlItemEnd 
\markdownRendererUlEnd \markdownRendererInterblockSeparator
{}\markdownRendererHeadingTwo{\markdownRendererLink{v7.3.2}{https://github.com/tuna/thuthesis/compare/v7.3.1...v7.3.2}{https://github.com/tuna/thuthesis/compare/v7.3.1...v7.3.2}{} - 2023-04-06}\markdownRendererInterblockSeparator
{}\markdownRendererHeadingThree{Fixed}\markdownRendererInterblockSeparator
{}\markdownRendererUlBegin
\markdownRendererUlItem 修复学术成果没有连续编号的问题(\markdownRendererLink{\markdownRendererHash{}825}{https://github.com/tuna/thuthesis/issues/825}{https://github.com/tuna/thuthesis/issues/825}{})。\markdownRendererUlItemEnd 
\markdownRendererUlItem 修复研究生个人简历部分行距过窄的问题 (\markdownRendererLink{\markdownRendererHash{}850}{https://github.com/tuna/thuthesis/issues/850}{https://github.com/tuna/thuthesis/issues/850}{})\markdownRendererUlItemEnd 
\markdownRendererUlEnd \markdownRendererInterblockSeparator
{}\markdownRendererHeadingThree{Changed}\markdownRendererInterblockSeparator
{}\markdownRendererUlBegin
\markdownRendererUlItem 修改部分选项的说明,同步《指南》2023 年 1、3 月版的更改。\markdownRendererUlItemEnd 
\markdownRendererUlEnd \markdownRendererInterblockSeparator
{}\markdownRendererHeadingTwo{\markdownRendererLink{v7.3.1}{https://github.com/tuna/thuthesis/compare/v7.3.0...v7.3.1}{https://github.com/tuna/thuthesis/compare/v7.3.0...v7.3.1}{} - 2022-10-05}\markdownRendererInterblockSeparator
{}\markdownRendererHeadingThree{Added}\markdownRendererInterblockSeparator
{}\markdownRendererUlBegin
\markdownRendererUlItem 增加选项 \markdownRendererCodeSpan{appendix-figure-in-lof} 控制附录中的图/表是否列入插图清单/附表清单。\markdownRendererUlItemEnd 
\markdownRendererUlEnd \markdownRendererInterblockSeparator
{}\markdownRendererHeadingThree{Changed}\markdownRendererInterblockSeparator
{}\markdownRendererUlBegin
\markdownRendererUlItem 修改“指导教师/小组评语”章节的名称,同步《指南》2022 年 9 月版的修改。\markdownRendererUlItemEnd 
\markdownRendererUlEnd \markdownRendererInterblockSeparator
{}\markdownRendererHeadingThree{Fixed}\markdownRendererInterblockSeparator
{}\markdownRendererUlBegin
\markdownRendererUlItem 修正答辩委员会名单页的行距。\markdownRendererUlItemEnd 
\markdownRendererUlItem 修复导言区中设置 \markdownRendererCodeSpan{toc-depth} 导致空白插图清单的 bug。\markdownRendererUlItemEnd 
\markdownRendererUlEnd \markdownRendererInterblockSeparator
{}\markdownRendererHeadingTwo{\markdownRendererLink{v7.3.0}{https://github.com/tuna/thuthesis/compare/v7.2.4...v7.3.0}{https://github.com/tuna/thuthesis/compare/v7.2.4...v7.3.0}{} - 2022-05-17}\markdownRendererInterblockSeparator
{}\markdownRendererHeadingThree{Added}\markdownRendererInterblockSeparator
{}\markdownRendererUlBegin
\markdownRendererUlItem 新增 LuaTeX 支持(试验性)(\markdownRendererLink{\markdownRendererHash{}771}{https://github.com/tuna/thuthesis/issues/771}{https://github.com/tuna/thuthesis/issues/771}{})。\markdownRendererUlItemEnd 
\markdownRendererUlEnd \markdownRendererInterblockSeparator
{}\markdownRendererHeadingThree{Changed}\markdownRendererInterblockSeparator
{}\markdownRendererUlBegin
\markdownRendererUlItem 研究生英文版目录中的标签分隔符由 \markdownRendererCodeSpan{\markdownRendererBackslash{}quad} 改为空格(\markdownRendererLink{\markdownRendererHash{}759}{https://github.com/tuna/thuthesis/discussions/759}{https://github.com/tuna/thuthesis/discussions/759}{})。\markdownRendererUlItemEnd 
\markdownRendererUlItem 研究生英文版章节标题的标签分隔符由 \markdownRendererCodeSpan{\markdownRendererBackslash{}quad} 改为空格(\markdownRendererLink{\markdownRendererHash{}759}{https://github.com/tuna/thuthesis/discussions/759}{https://github.com/tuna/thuthesis/discussions/759}{})。\markdownRendererUlItemEnd 
\markdownRendererUlItem 研究生英文封面的“Submitted”改为小写,同步《指南》2022 年 5 月版的修改。\markdownRendererUlItemEnd 
\markdownRendererUlItem 更改示例文档中 \markdownRendererCodeSpan{longtable} 的“续表”标题格式,同步《指南》2021 年 6 月版的修改(\markdownRendererLink{\markdownRendererHash{}766}{https://github.com/tuna/thuthesis/issues/766}{https://github.com/tuna/thuthesis/issues/766}{})。\markdownRendererUlItemEnd 
\markdownRendererUlEnd \markdownRendererInterblockSeparator
{}\markdownRendererHeadingThree{Fixed}\markdownRendererInterblockSeparator
{}\markdownRendererUlBegin
\markdownRendererUlItem 修正英文版研究成果的格式(\markdownRendererLink{\markdownRendererHash{}755}{https://github.com/tuna/thuthesis/issues/755}{https://github.com/tuna/thuthesis/issues/755}{})。\markdownRendererUlItemEnd 
\markdownRendererUlItem 修复脚注内容可能跨页的问题(\markdownRendererLink{\markdownRendererHash{}778}{https://github.com/tuna/thuthesis/issues/778}{https://github.com/tuna/thuthesis/issues/778}{})。\markdownRendererUlItemEnd 
\markdownRendererUlEnd \markdownRendererInterblockSeparator
{}\markdownRendererHeadingTwo{\markdownRendererLink{v7.2.4}{https://github.com/tuna/thuthesis/compare/v7.2.3...v7.2.4}{https://github.com/tuna/thuthesis/compare/v7.2.3...v7.2.4}{} - 2022-03-19}\markdownRendererInterblockSeparator
{}\markdownRendererHeadingThree{Added}\markdownRendererInterblockSeparator
{}\markdownRendererUlBegin
\markdownRendererUlItem 增加警告提醒本科生将附录置于声明后(\markdownRendererLink{\markdownRendererHash{}682}{https://github.com/tuna/thuthesis/issues/682}{https://github.com/tuna/thuthesis/issues/682}{})。\markdownRendererUlItemEnd 
\markdownRendererUlItem 增加警告提醒本科生将插图索引和表格索引置于正文后。\markdownRendererUlItemEnd 
\markdownRendererUlItem 本科生的 \markdownRendererCodeSpan{\markdownRendererBackslash{}listoffiguresandtables} 改为分开的插图索引和表格索引。\markdownRendererUlItemEnd 
\markdownRendererUlEnd \markdownRendererInterblockSeparator
{}\markdownRendererHeadingThree{Changed}\markdownRendererInterblockSeparator
{}\markdownRendererUlBegin
\markdownRendererUlItem 本科生的“答辩委员会名单”、“指导教师/小组学术评语”和“答辩委员会决议书”改为不输出内容(\markdownRendererLink{\markdownRendererHash{}688}{https://github.com/tuna/thuthesis/issues/688}{https://github.com/tuna/thuthesis/issues/688}{})。\markdownRendererUlItemEnd 
\markdownRendererUlItem 研究生和博后不再默认载入 \markdownRendererCodeSpan{bibunits}(\markdownRendererLink{\markdownRendererHash{}710}{https://github.com/tuna/thuthesis/issues/710}{https://github.com/tuna/thuthesis/issues/710}{})。\markdownRendererUlItemEnd 
\markdownRendererUlItem 参考文献表中预印本的文献类型标识改为“A”。\markdownRendererUlItemEnd 
\markdownRendererUlItem 允许多行页眉(\markdownRendererLink{\markdownRendererHash{}735}{https://github.com/tuna/thuthesis/issues/735}{https://github.com/tuna/thuthesis/issues/735}{})。\markdownRendererUlItemEnd 
\markdownRendererUlEnd \markdownRendererInterblockSeparator
{}\markdownRendererHeadingThree{Fixed}\markdownRendererInterblockSeparator
{}\markdownRendererUlBegin
\markdownRendererUlItem 修复本科生“使用授权说明”中“日期”后缺失的冒号(\markdownRendererLink{\markdownRendererHash{}679}{https://github.com/tuna/thuthesis/issues/679}{https://github.com/tuna/thuthesis/issues/679}{})。\markdownRendererUlItemEnd 
\markdownRendererUlItem 修复 \markdownRendererCodeSpan{TikZ} 的 \markdownRendererCodeSpan{external} 库与 \markdownRendererCodeSpan{pdfpages} 的兼容性问题(\markdownRendererLink{\markdownRendererHash{}693}{https://github.com/tuna/thuthesis/issues/693}{https://github.com/tuna/thuthesis/issues/693}{})。\markdownRendererUlItemEnd 
\markdownRendererUlItem 参考文献表中专利文献使用 \markdownRendererCodeSpan{address}/\markdownRendererCodeSpan{lcoation} 输出专利国别。\markdownRendererUlItemEnd 
\markdownRendererUlItem 修正一章内脚注数量超过 10 个时的报错问题,改为警告(\markdownRendererLink{\markdownRendererHash{}742}{https://github.com/tuna/thuthesis/issues/742}{https://github.com/tuna/thuthesis/issues/742}{})。\markdownRendererUlItemEnd 
\markdownRendererUlEnd \markdownRendererInterblockSeparator
{}\markdownRendererHeadingThree{Removed}\markdownRendererInterblockSeparator
{}\markdownRendererUlBegin
\markdownRendererUlItem 去掉了 \markdownRendererCodeSpan{siunitx} 的 \markdownRendererCodeSpan{inter-unit-product} 设置。\markdownRendererUlItemEnd 
\markdownRendererUlEnd \markdownRendererInterblockSeparator
{}\markdownRendererHeadingTwo{\markdownRendererLink{v7.2.3}{https://github.com/tuna/thuthesis/compare/v7.2.2...v7.2.3}{https://github.com/tuna/thuthesis/compare/v7.2.2...v7.2.3}{} - 2021-05-31}\markdownRendererInterblockSeparator
{}\markdownRendererHeadingThree{Changed}\markdownRendererInterblockSeparator
{}\markdownRendererUlBegin
\markdownRendererUlItem 中文模板的公式编号改为中文括号(\markdownRendererLink{\markdownRendererHash{}287}{https://github.com/tuna/thuthesis/issues/287}{https://github.com/tuna/thuthesis/issues/287}{})。\markdownRendererUlItemEnd 
\markdownRendererUlEnd \markdownRendererInterblockSeparator
{}\markdownRendererHeadingThree{Fixed}\markdownRendererInterblockSeparator
{}\markdownRendererUlBegin
\markdownRendererUlItem 修正硕士论文书脊的字号(\markdownRendererLink{\markdownRendererHash{}647}{https://github.com/tuna/thuthesis/issues/647}{https://github.com/tuna/thuthesis/issues/647}{})。\markdownRendererUlItemEnd 
\markdownRendererUlItem 修正本科生附录(调研和翻译)的目录在 TeX Live 2019 前无法生成的问题(\markdownRendererLink{\markdownRendererHash{}659}{https://github.com/tuna/thuthesis/issues/659}{https://github.com/tuna/thuthesis/issues/659}{})。\markdownRendererUlItemEnd 
\markdownRendererUlItem 修正本科生主要符号表的标题(\markdownRendererLink{\markdownRendererHash{}661}{https://github.com/tuna/thuthesis/issues/661}{https://github.com/tuna/thuthesis/issues/661}{})。\markdownRendererUlItemEnd 
\markdownRendererUlEnd \markdownRendererInterblockSeparator
{}\markdownRendererHeadingTwo{\markdownRendererLink{v7.2.2}{https://github.com/tuna/thuthesis/compare/v7.2.1...v7.2.2}{https://github.com/tuna/thuthesis/compare/v7.2.1...v7.2.2}{} - 2021-04-03}\markdownRendererInterblockSeparator
{}\markdownRendererHeadingThree{Changed}\markdownRendererInterblockSeparator
{}\markdownRendererUlBegin
\markdownRendererUlItem 修改授权说明的内容和格式,同 2020 年 12 月版 Word 模板一致(\markdownRendererLink{\markdownRendererHash{}625}{https://github.com/tuna/thuthesis/issues/625}{https://github.com/tuna/thuthesis/issues/625}{})。\markdownRendererUlItemEnd 
\markdownRendererUlItem 参考文献的页码前与冒号之间加上空格,同步 2021 年 3 月版《指南》的格式修改(\markdownRendererLink{\markdownRendererHash{}629}{https://github.com/tuna/thuthesis/issues/629}{https://github.com/tuna/thuthesis/issues/629}{})。\markdownRendererUlItemEnd 
\markdownRendererUlItem 著者-出版年制参考文献表的著者姓名与年份之间改为逗号。\markdownRendererUlItemEnd 
\markdownRendererUlEnd \markdownRendererInterblockSeparator
{}\markdownRendererHeadingThree{Fixed}\markdownRendererInterblockSeparator
{}\markdownRendererUlBegin
\markdownRendererUlItem 修正图表等浮动体与文字之间的距离(\markdownRendererLink{\markdownRendererHash{}614}{https://github.com/tuna/thuthesis/issues/614}{https://github.com/tuna/thuthesis/issues/614}{}、\markdownRendererLink{\markdownRendererHash{}617}{https://github.com/tuna/thuthesis/issues/617}{https://github.com/tuna/thuthesis/issues/617}{})。\markdownRendererUlItemEnd 
\markdownRendererUlItem 修正表格、算法等浮动体的行距(\markdownRendererLink{\markdownRendererHash{}619}{https://github.com/tuna/thuthesis/issues/619}{https://github.com/tuna/thuthesis/issues/619}{})。\markdownRendererUlItemEnd 
\markdownRendererUlItem 修正了上标式引用后与中文之间多余的空格(\markdownRendererLink{\markdownRendererHash{}624}{https://github.com/tuna/thuthesis/issues/624}{https://github.com/tuna/thuthesis/issues/624}{})。\markdownRendererUlItemEnd 
\markdownRendererUlItem 修正了参考文献的姓名或年份中含有中括号时的引用错误(\markdownRendererLink{\markdownRendererHash{}630}{https://github.com/tuna/thuthesis/issues/630}{https://github.com/tuna/thuthesis/issues/630}{})。\markdownRendererUlItemEnd 
\markdownRendererUlEnd \markdownRendererInterblockSeparator
{}\markdownRendererHeadingTwo{\markdownRendererLink{v7.2.1}{https://github.com/tuna/thuthesis/compare/v7.2.0...v7.2.1}{https://github.com/tuna/thuthesis/compare/v7.2.0...v7.2.1}{} - 2021-03-21}\markdownRendererInterblockSeparator
{}\markdownRendererHeadingThree{Added}\markdownRendererInterblockSeparator
{}\markdownRendererUlBegin
\markdownRendererUlItem 在文档中添加更多关于数学公式样式的说明。\markdownRendererUlItemEnd 
\markdownRendererUlEnd \markdownRendererInterblockSeparator
{}\markdownRendererHeadingThree{Changed}\markdownRendererInterblockSeparator
{}\markdownRendererUlBegin
\markdownRendererUlItem 允许控制研究生的声明页是否添加页眉页脚。\markdownRendererUlItemEnd 
\markdownRendererUlEnd \markdownRendererInterblockSeparator
{}\markdownRendererHeadingThree{Fixed}\markdownRendererInterblockSeparator
{}\markdownRendererUlBegin
\markdownRendererUlItem 调整文字与图表等浮动体之间的距离(\markdownRendererLink{\markdownRendererHash{}614}{https://github.com/tuna/thuthesis/issues/614}{https://github.com/tuna/thuthesis/issues/614}{})。\markdownRendererUlItemEnd 
\markdownRendererUlItem 修复一些字体选择相关的问题。\markdownRendererUlItemEnd 
\markdownRendererUlEnd \markdownRendererInterblockSeparator
{}\markdownRendererHeadingTwo{\markdownRendererLink{v7.2.0}{https://github.com/tuna/thuthesis/compare/v7.1.0...v7.2.0}{https://github.com/tuna/thuthesis/compare/v7.1.0...v7.2.0}{} - 2021-03-12}\markdownRendererInterblockSeparator
{}\markdownRendererHeadingThree{Added}\markdownRendererInterblockSeparator
{}\markdownRendererUlBegin
\markdownRendererUlItem 新增英文版写作指南要求的格式。\markdownRendererUlItemEnd 
\markdownRendererUlItem 新增选题报告的格式(\markdownRendererLink{\markdownRendererHash{}579}{https://github.com/tuna/thuthesis/issues/579}{https://github.com/tuna/thuthesis/issues/579}{})。\markdownRendererUlItemEnd 
\markdownRendererUlItem 新增 \markdownRendererCodeSpan{figure-number-sepatator} 等选项设置图表编号的连接符。\markdownRendererUlItemEnd 
\markdownRendererUlItem 新增数学符号字体风格选项 \markdownRendererCodeSpan{math-style}。\markdownRendererUlItemEnd 
\markdownRendererUlItem 新增选项控制数学字体风格的细节:\markdownRendererCodeSpan{uppercase-greek}、\markdownRendererCodeSpan{less-than-or-equal}、\markdownRendererCodeSpan{integral}、\markdownRendererCodeSpan{integral-limits}、\markdownRendererCodeSpan{partial} 和 \markdownRendererCodeSpan{math-ellipsis}。\markdownRendererUlItemEnd 
\markdownRendererUlItem 新增数学字体试验性选项 \markdownRendererCodeSpan{math-font = newtx}。\markdownRendererUlItemEnd 
\markdownRendererUlEnd \markdownRendererInterblockSeparator
{}\markdownRendererHeadingThree{Changed}\markdownRendererInterblockSeparator
{}\markdownRendererUlBegin
\markdownRendererUlItem 研究生的声明页默认加上页眉和页码,不受 \markdownRendererCodeSpan{page-style} 的控制(\markdownRendererLink{\markdownRendererHash{}574}{https://github.com/tuna/thuthesis/issues/574}{https://github.com/tuna/thuthesis/issues/574}{})。\markdownRendererUlItemEnd 
\markdownRendererUlItem 取消图表标题的悬挂缩进(\markdownRendererLink{\markdownRendererHash{}589}{https://github.com/tuna/thuthesis/issues/589}{https://github.com/tuna/thuthesis/issues/589}{})。\markdownRendererUlItemEnd 
\markdownRendererUlItem 英文封面的联合导师改为“Co-supervisor”。\markdownRendererUlItemEnd 
\markdownRendererUlItem 联合导师的 key 改为 \markdownRendererCodeSpan{co-supervisor},同英文版模板一致。\markdownRendererUlItemEnd 
\markdownRendererUlEnd \markdownRendererInterblockSeparator
{}\markdownRendererHeadingThree{Fixed}\markdownRendererInterblockSeparator
{}\markdownRendererUlBegin
\markdownRendererUlItem 修正 \markdownRendererCodeSpan{longtable} 宏包的配置(\markdownRendererLink{\markdownRendererHash{}584}{https://github.com/tuna/thuthesis/issues/584}{https://github.com/tuna/thuthesis/issues/584}{})。\markdownRendererUlItemEnd 
\markdownRendererUlItem 修正本科生的“目录”、“声明”和“致谢”等标题中的空白(\markdownRendererLink{\markdownRendererHash{}591}{https://github.com/tuna/thuthesis/issues/591}{https://github.com/tuna/thuthesis/issues/591}{})。\markdownRendererUlItemEnd 
\markdownRendererUlItem 修正参考文献的格式,取消页码与前面冒号之间的空格。\markdownRendererUlItemEnd 
\markdownRendererUlItem 修正中文封面的字距。\markdownRendererUlItemEnd 
\markdownRendererUlEnd \markdownRendererInterblockSeparator
{}\markdownRendererHeadingTwo{\markdownRendererLink{v7.1.0}{https://github.com/tuna/thuthesis/compare/v7.0.0...v7.1.0}{https://github.com/tuna/thuthesis/compare/v7.0.0...v7.1.0}{} - 2020-10-14}\markdownRendererInterblockSeparator
{}\markdownRendererHeadingThree{Changed}\markdownRendererInterblockSeparator
{}\markdownRendererUlBegin
\markdownRendererUlItem 更新摘要的标题格式(研究生 2020-09-18 版)。\markdownRendererUlItemEnd 
\markdownRendererUlItem 更新目录的格式(研究生 2020-09-18 版)。\markdownRendererUlItemEnd 
\markdownRendererUlItem 图表浮动体的位置参数默认为 \markdownRendererCodeSpan{h}。\markdownRendererUlItemEnd 
\markdownRendererUlItem 更新示例文档。\markdownRendererUlItemEnd 
\markdownRendererUlEnd \markdownRendererInterblockSeparator
{}\markdownRendererHeadingThree{Fixed}\markdownRendererInterblockSeparator
{}\markdownRendererUlBegin
\markdownRendererUlItem 修正 “keywords” 的拼写。\markdownRendererUlItemEnd 
\markdownRendererUlItem 修正授权使用说明的内容。\markdownRendererUlItemEnd 
\markdownRendererUlItem 修正授伪粗字体的粗度。\markdownRendererUlItemEnd 
\markdownRendererUlItem 修正 \markdownRendererCodeSpan{\markdownRendererBackslash{}small} 等字号命令的行距。\markdownRendererUlItemEnd 
\markdownRendererUlItem 修正数学公式前后的距离。\markdownRendererUlItemEnd 
\markdownRendererUlItem 修正个人简历和学术成果的格式。\markdownRendererUlItemEnd 
\markdownRendererUlItem 修正图表标题的行距。\markdownRendererUlItemEnd 
\markdownRendererUlItem 禁止同一条参考文献中间分页。\markdownRendererUlItemEnd 
\markdownRendererUlItem 修正脚注的行距和缩进距离。\markdownRendererUlItemEnd 
\markdownRendererUlEnd \markdownRendererInterblockSeparator
{}\markdownRendererHeadingTwo{\markdownRendererLink{v7.0.0}{https://github.com/tuna/thuthesis/compare/v6.1.3...v7.0.0}{https://github.com/tuna/thuthesis/compare/v6.1.3...v7.0.0}{} - 2020-09-09}\markdownRendererInterblockSeparator
{}\markdownRendererHeadingThree{Changed}\markdownRendererInterblockSeparator
{}\markdownRendererUlBegin
\markdownRendererUlItem 更新 2020 年版目录、插图和附表清单的格式。\markdownRendererUlItemEnd 
\markdownRendererUlItem 更新 2020 年版使用授权说明的内容。\markdownRendererUlItemEnd 
\markdownRendererUlItem 更新 2020 年版参考文献表的格式。\markdownRendererUlItemEnd 
\markdownRendererUlEnd \markdownRendererInterblockSeparator
{}\markdownRendererHeadingThree{Added}\markdownRendererInterblockSeparator
{}\markdownRendererUlBegin
\markdownRendererUlItem 新增 2020 年版“答辩委员会名单”页。\markdownRendererUlItemEnd 
\markdownRendererUlItem 新增 biblatex 支持。\markdownRendererUlItemEnd 
\markdownRendererUlItem 新增本科生外文系格式的支持。\markdownRendererUlItemEnd 
\markdownRendererUlEnd \markdownRendererInterblockSeparator
{}\markdownRendererHeadingThree{Fixed}\markdownRendererInterblockSeparator
{}\markdownRendererUlBegin
\markdownRendererUlItem 修正本科生密级的字体。\markdownRendererUlItemEnd 
\markdownRendererUlItem 修正表格的默认字号。\markdownRendererUlItemEnd 
\markdownRendererUlItem 修正参考文献表的行距和段前段后间距。\markdownRendererUlItemEnd 
\markdownRendererUlItem 修正 \markdownRendererCodeSpan{\markdownRendererBackslash{}citep} 数字式引用的页码位置。\markdownRendererUlItemEnd 
\markdownRendererUlItem 修改摘要中关键词的格式。\markdownRendererUlItemEnd 
\markdownRendererUlItem 修正封面的语言切换。\markdownRendererUlItemEnd 
\markdownRendererUlEnd \markdownRendererInterblockSeparator
{}\markdownRendererHeadingTwo{\markdownRendererLink{v6.1.3}{https://github.com/tuna/thuthesis/compare/v6.1.2...v6.1.3}{https://github.com/tuna/thuthesis/compare/v6.1.2...v6.1.3}{} - 2020-07-09}\markdownRendererInterblockSeparator
{}\markdownRendererHeadingThree{Added}\markdownRendererInterblockSeparator
{}\markdownRendererUlBegin
\markdownRendererUlItem 新增选项 \markdownRendererCodeSpan{statement-page-style = empty / plain} 同时控制声明的页眉和页脚。\markdownRendererUlItemEnd 
\markdownRendererUlEnd \markdownRendererInterblockSeparator
{}\markdownRendererHeadingThree{Fixed}\markdownRendererInterblockSeparator
{}\markdownRendererUlBegin
\markdownRendererUlItem \markdownRendererCodeSpan{\markdownRendererBackslash{}record} 命令中,如果 \markdownRendererCodeSpan{output} 配置为 \markdownRendererCodeSpan{print},则强制进行 \markdownRendererCodeSpan{\markdownRendererBackslash{}cleardoublepage},保证记录表独立成页。\markdownRendererUlItemEnd 
\markdownRendererUlItem 修正了在导言区设置论文主要语言无效的问题(\markdownRendererLink{\markdownRendererHash{}560}{https://github.com/tuna/thuthesis/issues/560}{https://github.com/tuna/thuthesis/issues/560}{})。\markdownRendererUlItemEnd 
\markdownRendererUlItem 修正了研究生插入扫描版声明页时的页眉。\markdownRendererUlItemEnd 
\markdownRendererUlEnd \markdownRendererInterblockSeparator
{}\markdownRendererHeadingThree{Deprecated}\markdownRendererInterblockSeparator
{}\markdownRendererUlBegin
\markdownRendererUlItem 选项 \markdownRendererCodeSpan{statement-page-number} 已过时。\markdownRendererUlItemEnd 
\markdownRendererUlEnd \markdownRendererInterblockSeparator
{}\markdownRendererHeadingTwo{\markdownRendererLink{v6.1.2}{https://github.com/tuna/thuthesis/compare/v6.1.1...v6.1.2}{https://github.com/tuna/thuthesis/compare/v6.1.1...v6.1.2}{} - 2020-06-14}\markdownRendererInterblockSeparator
{}\markdownRendererHeadingThree{Changed}\markdownRendererInterblockSeparator
{}\markdownRendererUlBegin
\markdownRendererUlItem \markdownRendererCodeSpan{\markdownRendererBackslash{}statement} 和 \markdownRendererCodeSpan{\markdownRendererBackslash{}copyrightpage} 命令都会在 PDF 中生成相应位置的书签。\markdownRendererUlItemEnd 
\markdownRendererUlItem \markdownRendererCodeSpan{\markdownRendererBackslash{}statement} 命令编译生成声明页默认不含页码。\markdownRendererUlItemEnd 
\markdownRendererUlItem \markdownRendererCodeSpan{\markdownRendererBackslash{}statement[xxx.pdf]} 插入扫描页时在页脚生成页码,以解决打印版与电子版页码不一致的情况。\markdownRendererUlItemEnd 
\markdownRendererUlItem 使用 \markdownRendererCodeSpan{l3build} 编译的版本,在提交到 CTAN 时提供 TDS 结构,以解决编译时 \markdownRendererCodeSpan{tsinghua-name-bachelor.pdf} 找不到的问题。\markdownRendererUlItemEnd 
\markdownRendererUlEnd \markdownRendererInterblockSeparator
{}\markdownRendererHeadingThree{Added}\markdownRendererInterblockSeparator
{}\markdownRendererUlBegin
\markdownRendererUlItem 添加 \markdownRendererCodeSpan{\markdownRendererBackslash{}record} 命令用于本科生插入综合论文训练记录表,同时在 PDF 中生成对应书签。\markdownRendererUlItemEnd 
\markdownRendererUlItem 添加选项 \markdownRendererCodeSpan{statement-page-number} 控制编译声明页时是否含页码。\markdownRendererUlItemEnd 
\markdownRendererUlEnd \markdownRendererInterblockSeparator
{}\markdownRendererHeadingThree{Fixed}\markdownRendererInterblockSeparator
{}\markdownRendererUlBegin
\markdownRendererUlItem 修正示例代码中关于 \markdownRendererCodeSpan{\markdownRendererBackslash{}statement} 的 typo。\markdownRendererUlItemEnd 
\markdownRendererUlEnd \markdownRendererInterblockSeparator
{}\markdownRendererHeadingTwo{\markdownRendererLink{v6.1.1}{https://github.com/tuna/thuthesis/compare/v6.1.0...v6.1.1}{https://github.com/tuna/thuthesis/compare/v6.1.0...v6.1.1}{} - 2020-06-12}\markdownRendererInterblockSeparator
{}\markdownRendererHeadingThree{Changed}\markdownRendererInterblockSeparator
{}\markdownRendererUlBegin
\markdownRendererUlItem 书脊中的西文不再需要调整高度。\markdownRendererUlItemEnd 
\markdownRendererUlItem 修改预生成的隶书版本本科生封面学校名称的文件名为 \markdownRendererCodeSpan{tsinghua-name-bachelor.pdf},不再尝试使用系统字体生成。\markdownRendererUlItemEnd 
\markdownRendererUlItem 将论文示例的文件名更改为 \markdownRendererCodeSpan{thuthesis-example.tex / pdf},以符合 CTAN 的要求。\markdownRendererUlItemEnd 
\markdownRendererUlEnd \markdownRendererInterblockSeparator
{}\markdownRendererHeadingThree{Added}\markdownRendererInterblockSeparator
{}\markdownRendererUlBegin
\markdownRendererUlItem 添加选项 \markdownRendererCodeSpan{include-spine},允许在正文中插入书脊页(\markdownRendererLink{\markdownRendererHash{}551}{https://github.com/tuna/thuthesis/issues/551}{https://github.com/tuna/thuthesis/issues/551}{})。\markdownRendererUlItemEnd 
\markdownRendererUlItem 添加选项 \markdownRendererCodeSpan{spine-title}、\markdownRendererCodeSpan{spine-author} 控制书脊的内容。\markdownRendererUlItemEnd 
\markdownRendererUlItem 添加选项 \markdownRendererCodeSpan{spine-font} 控制书脊的字号。\markdownRendererUlItemEnd 
\markdownRendererUlItem 添加选项 \markdownRendererCodeSpan{output} 选择输出格式为打印版或用于提交的电子版(\markdownRendererLink{\markdownRendererHash{}553}{https://github.com/tuna/thuthesis/issues/553}{https://github.com/tuna/thuthesis/issues/553}{})。\markdownRendererUlItemEnd 
\markdownRendererUlEnd \markdownRendererInterblockSeparator
{}\markdownRendererHeadingThree{Fixed}\markdownRendererInterblockSeparator
{}\markdownRendererUlBegin
\markdownRendererUlItem 修正书脊的格式。\markdownRendererUlItemEnd 
\markdownRendererUlItem 修复文档中 Changelog 的格式问题。\markdownRendererUlItemEnd 
\markdownRendererUlEnd \markdownRendererInterblockSeparator
{}\markdownRendererHeadingTwo{\markdownRendererLink{v6.1.0}{https://github.com/tuna/thuthesis/compare/v6.0.2...v6.1.0}{https://github.com/tuna/thuthesis/compare/v6.0.2...v6.1.0}{} - 2020-06-08}\markdownRendererInterblockSeparator
{}\markdownRendererHeadingThree{Changed}\markdownRendererInterblockSeparator
{}\markdownRendererUlBegin
\markdownRendererUlItem 在 \markdownRendererCodeSpan{translation} 环境中使用 \markdownRendererCodeSpan{\markdownRendererBackslash{}bibliography} 改为生成参考文献, 对应的原文索引改为 \markdownRendererCodeSpan{translation-index} 环境(\markdownRendererLink{\markdownRendererHash{}529}{https://github.com/tuna/thuthesis/issues/529}{https://github.com/tuna/thuthesis/issues/529}{})。\markdownRendererUlItemEnd 
\markdownRendererUlItem 附录的图、表不再加入索引。\markdownRendererUlItemEnd 
\markdownRendererUlItem 使用 \markdownRendererCodeSpan{threeparttable} 示例表内脚注。\markdownRendererUlItemEnd 
\markdownRendererUlItem 本科生的目录章标题的西文字母和数字默认使用 Arial(\markdownRendererLink{\markdownRendererHash{}542}{https://github.com/tuna/thuthesis/issues/542}{https://github.com/tuna/thuthesis/issues/542}{})。\markdownRendererUlItemEnd 
\markdownRendererUlItem GitHub repo 所有者更改为清华大学 TUNA 协会。\markdownRendererUlItemEnd 
\markdownRendererUlItem 接管 \markdownRendererCodeSpan{ctex} 的 \markdownRendererCodeSpan{fontset} 选项,允许更灵活的字体配置(\markdownRendererLink{\markdownRendererHash{}498}{https://github.com/tuna/thuthesis/issues/498}{https://github.com/tuna/thuthesis/issues/498}{})。\markdownRendererUlItemEnd 
\markdownRendererUlEnd \markdownRendererInterblockSeparator
{}\markdownRendererHeadingThree{Fixed}\markdownRendererInterblockSeparator
{}\markdownRendererUlBegin
\markdownRendererUlItem 本科生附录的调研报告使用英文(\markdownRendererLink{\markdownRendererHash{}479}{https://github.com/tuna/thuthesis/issues/479}{https://github.com/tuna/thuthesis/issues/479}{})。\markdownRendererUlItemEnd 
\markdownRendererUlItem 修正本科生附录的 \markdownRendererCodeSpan{algorithm} 和 \markdownRendererCodeSpan{listings} 环境的编号格式。\markdownRendererUlItemEnd 
\markdownRendererUlItem 研究生的“使用授权说明”增加一空白页(\markdownRendererLink{\markdownRendererHash{}504}{https://github.com/tuna/thuthesis/issues/504}{https://github.com/tuna/thuthesis/issues/504}{})。\markdownRendererUlItemEnd 
\markdownRendererUlItem 修正 publication 列表行距问题(\markdownRendererLink{\markdownRendererHash{}507}{https://github.com/tuna/thuthesis/issues/507}{https://github.com/tuna/thuthesis/issues/507}{})。\markdownRendererUlItemEnd 
\markdownRendererUlItem 修正研究生目录的行距。\markdownRendererUlItemEnd 
\markdownRendererUlItem 调整本科生封面有辅导教师、联合指导教师时的格式(\markdownRendererLink{\markdownRendererHash{}522}{https://github.com/tuna/thuthesis/issues/522}{https://github.com/tuna/thuthesis/issues/522}{}, \markdownRendererLink{\markdownRendererHash{}537}{https://github.com/tuna/thuthesis/issues/537}{https://github.com/tuna/thuthesis/issues/537}{})。\markdownRendererUlItemEnd 
\markdownRendererUlItem 修正本科生封面日期的字体(\markdownRendererLink{\markdownRendererHash{}532}{https://github.com/tuna/thuthesis/issues/532}{https://github.com/tuna/thuthesis/issues/532}{})。\markdownRendererUlItemEnd 
\markdownRendererUlItem 设置三线表横线的默认粗度。\markdownRendererUlItemEnd 
\markdownRendererUlItem 修正文档中关于本科生学科(专业)名称的说明。\markdownRendererUlItemEnd 
\markdownRendererUlItem 允许用户使用文档类选项 \markdownRendererCodeSpan{openright}。\markdownRendererUlItemEnd 
\markdownRendererUlItem 修正本科生的“单设一页”格式(\markdownRendererLink{\markdownRendererHash{}545}{https://github.com/tuna/thuthesis/issues/545}{https://github.com/tuna/thuthesis/issues/545}{})。\markdownRendererUlItemEnd 
\markdownRendererUlItem 研究生的正文(main matter)起始于奇数页。\markdownRendererUlItemEnd 
\markdownRendererUlEnd \markdownRendererInterblockSeparator
{}\markdownRendererHeadingThree{Added}\markdownRendererInterblockSeparator
{}\markdownRendererUlBegin
\markdownRendererUlItem 允许使用边注。\markdownRendererUlItemEnd 
\markdownRendererUlItem 新增 \markdownRendererCodeSpan{amsthm} 支持。\markdownRendererUlItemEnd 
\markdownRendererUlItem 支持单个关键词设置样式。\markdownRendererUlItemEnd 
\markdownRendererUlItem 在 \markdownRendererCodeSpan{survey} 和 \markdownRendererCodeSpan{translation} 环境中生成独立目录。\markdownRendererUlItemEnd 
\markdownRendererUlItem 添加选项 \markdownRendererCodeSpan{toc-chapter-style} 控制本科生的目录章标题的西文字体。\markdownRendererUlItemEnd 
\markdownRendererUlEnd \markdownRendererInterblockSeparator
{}\markdownRendererHeadingTwo{\markdownRendererLink{v6.0.2}{https://github.com/tuna/thuthesis/compare/v6.0.1...v6.0.2}{https://github.com/tuna/thuthesis/compare/v6.0.1...v6.0.2}{} - 2020-02-23}\markdownRendererInterblockSeparator
{}\markdownRendererHeadingThree{Fixed}\markdownRendererInterblockSeparator
{}\markdownRendererUlBegin
\markdownRendererUlItem 修复图表公式目录内容缺失(\markdownRendererLink{\markdownRendererHash{}467}{https://github.com/tuna/thuthesis/issues/467}{https://github.com/tuna/thuthesis/issues/467}{})。\markdownRendererUlItemEnd 
\markdownRendererUlItem 修复 Github Actions 执行 l3build check 问题。\markdownRendererUlItemEnd 
\markdownRendererUlItem 文本模式使用 \markdownRendererCodeSpan{\markdownRendererBackslash{}checkmark}。\markdownRendererUlItemEnd 
\markdownRendererUlEnd \markdownRendererInterblockSeparator
{}\markdownRendererHeadingThree{Added}\markdownRendererInterblockSeparator
{}\markdownRendererUlBegin
\markdownRendererUlItem 补充 PDF 元信息:文档标题,语言。\markdownRendererUlItemEnd 
\markdownRendererUlEnd \markdownRendererInterblockSeparator
{}\markdownRendererHeadingThree{Changed}\markdownRendererInterblockSeparator
{}\markdownRendererUlBegin
\markdownRendererUlItem 补充“著者-出版年”引用命令使用方法。\markdownRendererUlItemEnd 
\markdownRendererUlItem 使用 \markdownRendererCodeSpan{xeCJKfntef} 替换 \markdownRendererCodeSpan{CJKfntef}。\markdownRendererUlItemEnd 
\markdownRendererUlEnd \markdownRendererInterblockSeparator
{}\markdownRendererHeadingTwo{\markdownRendererLink{v6.0.1}{https://github.com/tuna/thuthesis/compare/v6.0.0...v6.0.1}{https://github.com/tuna/thuthesis/compare/v6.0.0...v6.0.1}{} - 2020-02-03}\markdownRendererInterblockSeparator
{}\markdownRendererHeadingThree{Changed}\markdownRendererInterblockSeparator
{}\markdownRendererUlBegin
\markdownRendererUlItem 更新文档。\markdownRendererUlItemEnd 
\markdownRendererUlItem 更新 bib 测试用例。\markdownRendererUlItemEnd 
\markdownRendererUlEnd \markdownRendererInterblockSeparator
{}\markdownRendererHeadingThree{Fixed}\markdownRendererInterblockSeparator
{}\markdownRendererUlBegin
\markdownRendererUlItem 修复 PDF 目录层级问题(\markdownRendererLink{\markdownRendererHash{}457}{https://github.com/tuna/thuthesis/issues/457}{https://github.com/tuna/thuthesis/issues/457}{})。\markdownRendererUlItemEnd 
\markdownRendererUlItem 修复 PDF 中章节连接问题(\markdownRendererLink{\markdownRendererHash{}453}{https://github.com/tuna/thuthesis/issues/453}{https://github.com/tuna/thuthesis/issues/453}{})。\markdownRendererUlItemEnd 
\markdownRendererUlEnd \markdownRendererInterblockSeparator
{}\markdownRendererHeadingTwo{\markdownRendererLink{v6.0.0}{https://github.com/tuna/thuthesis/compare/v5.5.2...v6.0.0}{https://github.com/tuna/thuthesis/compare/v5.5.2...v6.0.0}{} - 2020-01-06}\markdownRendererInterblockSeparator
{}\markdownRendererHeadingThree{Added}\markdownRendererInterblockSeparator
{}\markdownRendererUlBegin
\markdownRendererUlItem 新增 \markdownRendererCodeSpan{survey}(调研阅读报告)和 \markdownRendererCodeSpan{translation}(书面翻译)环境用于本科生的附录, 其中的参考文献引用独立于论文正文(\markdownRendererLink{\markdownRendererHash{}343}{https://github.com/tuna/thuthesis/issues/343}{https://github.com/tuna/thuthesis/issues/343}{})。\markdownRendererUlItemEnd 
\markdownRendererUlItem 新增论文语言选项。\markdownRendererUlItemEnd 
\markdownRendererUlItem 增加 \markdownRendererCodeSpan{l3build} 测试。\markdownRendererUlItemEnd 
\markdownRendererUlEnd \markdownRendererInterblockSeparator
{}\markdownRendererHeadingThree{Changed}\markdownRendererInterblockSeparator
{}\markdownRendererUlBegin
\markdownRendererUlItem 重新设计 \markdownRendererCodeSpan{\markdownRendererBackslash{}thusetup} 的接口。\markdownRendererUlItemEnd 
\markdownRendererUlItem 指导教师的姓名和职称用英文逗号分开(\markdownRendererLink{\markdownRendererHash{}424}{https://github.com/tuna/thuthesis/issues/424}{https://github.com/tuna/thuthesis/issues/424}{})。\markdownRendererUlItemEnd 
\markdownRendererUlItem 重新设计封面(题名页)。\markdownRendererUlItemEnd 
\markdownRendererUlItem \markdownRendererCodeSpan{\markdownRendererBackslash{}makecover} 拆分为 \markdownRendererCodeSpan{\markdownRendererBackslash{}maketitle}、\markdownRendererCodeSpan{\markdownRendererBackslash{}copyrightpage}。\markdownRendererUlItemEnd 
\markdownRendererUlItem 中英文摘分别用 \markdownRendererCodeSpan{abstract}、\markdownRendererCodeSpan{abstract*} 环境生成。\markdownRendererUlItemEnd 
\markdownRendererUlItem 附录使用 \markdownRendererCodeSpan{\markdownRendererBackslash{}appendix} 命令开始,不再使用 \markdownRendererCodeSpan{appendix} 环境。\markdownRendererUlItemEnd 
\markdownRendererUlItem 修改 \markdownRendererCodeSpan{shuji} 为 \markdownRendererCodeSpan{spine}。\markdownRendererUlItemEnd 
\markdownRendererUlItem 修改 \markdownRendererCodeSpan{acknowledgement} 为 \markdownRendererCodeSpan{acknowledgements}。\markdownRendererUlItemEnd 
\markdownRendererUlItem 从 travis 切换到 github actions。\markdownRendererUlItemEnd 
\markdownRendererUlItem 更改日志从 \markdownRendererCodeSpan{thuthesis.dtx} 挪到 \markdownRendererCodeSpan{CHANGELOG.md}。\markdownRendererUlItemEnd 
\markdownRendererUlItem 整理 Makefile,便于日常使用。\markdownRendererUlItemEnd 
\markdownRendererUlEnd \markdownRendererInterblockSeparator
{}\markdownRendererHeadingTwo{Removed}\markdownRendererInterblockSeparator
{}\markdownRendererUlBegin
\markdownRendererUlItem 移除 \markdownRendererCodeSpan{secret} 选项。\markdownRendererUlItemEnd 
\markdownRendererUlItem 移除 \markdownRendererCodeSpan{translationbib} 环境。\markdownRendererUlItemEnd 
\markdownRendererUlItem 移除 \markdownRendererCodeSpan{tocarialchapter}、\markdownRendererCodeSpan{tocarialchapterentry} 和 \markdownRendererCodeSpan{tocarialchapterpage} 选项。\markdownRendererUlItemEnd 
\markdownRendererUlItem 只保留 xelatex 编译方式。\markdownRendererUlItemEnd 
\markdownRendererUlEnd \markdownRendererInterblockSeparator
{}\markdownRendererHeadingThree{Fixed}\markdownRendererInterblockSeparator
{}\markdownRendererUlBegin
\markdownRendererUlItem 修正本科生的文献引用样式。\markdownRendererUlItemEnd 
\markdownRendererUlItem 修正本科生目录章标题的字体。\markdownRendererUlItemEnd 
\markdownRendererUlItem 处理 \markdownRendererCodeSpan{hyperref} 与 \markdownRendererCodeSpan{unicode-math} 的兼容性问题。\markdownRendererUlItemEnd 
\markdownRendererUlEnd \markdownRendererInterblockSeparator
{}\markdownRendererHeadingTwo{\markdownRendererLink{v5.5.2}{https://github.com/tuna/thuthesis/compare/v5.5.1...v5.5.2}{https://github.com/tuna/thuthesis/compare/v5.5.1...v5.5.2}{} - 2019-04-21}\markdownRendererInterblockSeparator
{}\markdownRendererHeadingThree{Changed}\markdownRendererInterblockSeparator
{}\markdownRendererUlBegin
\markdownRendererUlItem 使用 XITS 数学字体。\markdownRendererUlItemEnd 
\markdownRendererUlEnd \markdownRendererInterblockSeparator
{}\markdownRendererHeadingTwo{\markdownRendererLink{v5.5.1}{https://github.com/tuna/thuthesis/compare/v5.5.0...v5.5.1}{https://github.com/tuna/thuthesis/compare/v5.5.0...v5.5.1}{} - 2019-04-16}\markdownRendererInterblockSeparator
{}\markdownRendererHeadingThree{Changed}\markdownRendererInterblockSeparator
{}\markdownRendererUlBegin
\markdownRendererUlItem \markdownRendererCodeSpan{\markdownRendererBackslash{}thu@textcircled}: 修复 minipage 中 footnote 编号问题。\markdownRendererUlItemEnd 
\markdownRendererUlEnd \markdownRendererInterblockSeparator
{}\markdownRendererHeadingTwo{\markdownRendererLink{v5.5.0}{https://github.com/tuna/thuthesis/compare/v5.4.5...v5.5.0}{https://github.com/tuna/thuthesis/compare/v5.4.5...v5.5.0}{} - 2019-03-15}\markdownRendererInterblockSeparator
{}\markdownRendererHeadingThree{Changed}\markdownRendererInterblockSeparator
{}\markdownRendererUlBegin
\markdownRendererUlItem 增加选项使用英文模板。\markdownRendererUlItemEnd 
\markdownRendererUlItem 使用 \markdownRendererCodeSpan{unicode-math} 处理数学字体。\markdownRendererUlItemEnd 
\markdownRendererUlItem Windows 的中文字体开启伪粗。\markdownRendererUlItemEnd 
\markdownRendererUlItem \markdownRendererCodeSpan{\markdownRendererBackslash{}thu@textcircled}: 去掉 \markdownRendererCodeSpan{pifootnote} 选项。\markdownRendererUlItemEnd 
\markdownRendererUlItem 移除 cfg 文件。\markdownRendererUlItemEnd 
\markdownRendererUlItem 修正图表标题与文字之间的距离。\markdownRendererUlItemEnd 
\markdownRendererUlItem \markdownRendererCodeSpan{\markdownRendererBackslash{}thu@first@titlepage}: 修正博士后封面的格式。\markdownRendererUlItemEnd 
\markdownRendererUlItem 增加 \markdownRendererCodeSpan{nomencl} 宏包的支持。\markdownRendererUlItemEnd 
\markdownRendererUlEnd \markdownRendererInterblockSeparator
{}\markdownRendererHeadingTwo{\markdownRendererLink{v5.4.5}{https://github.com/tuna/thuthesis/compare/v5.4.4...v5.4.5}{https://github.com/tuna/thuthesis/compare/v5.4.4...v5.4.5}{} - 2018-05-17}\markdownRendererInterblockSeparator
{}\markdownRendererHeadingThree{Changed}\markdownRendererInterblockSeparator
{}\markdownRendererUlBegin
\markdownRendererUlItem \markdownRendererCodeSpan{\markdownRendererBackslash{}normalsize}: 调整公式和正文间距。\markdownRendererUlItemEnd 
\markdownRendererUlEnd \markdownRendererInterblockSeparator
{}\markdownRendererHeadingTwo{\markdownRendererLink{v5.4.4}{https://github.com/tuna/thuthesis/compare/v5.4.2...v5.4.4}{https://github.com/tuna/thuthesis/compare/v5.4.2...v5.4.4}{} - 2018-04-22}\markdownRendererInterblockSeparator
{}\markdownRendererHeadingThree{Changed}\markdownRendererInterblockSeparator
{}\markdownRendererUlBegin
\markdownRendererUlItem 删除 \markdownRendererCodeSpan{arialtitle},所有论文格式都一样。\markdownRendererUlItemEnd 
\markdownRendererUlItem 让目录中的引用不影响正文中引用序号。\markdownRendererUlItemEnd 
\markdownRendererUlItem 参考文献列表的页码使用 hyphen 取代 en dash。\markdownRendererUlItemEnd 
\markdownRendererUlItem 参考文献标号左对齐。\markdownRendererUlItemEnd 
\markdownRendererUlItem 允许连续两个文献引用使用连接号。\markdownRendererUlItemEnd 
\markdownRendererUlEnd \markdownRendererInterblockSeparator
{}\markdownRendererHeadingTwo{\markdownRendererLink{v5.4.2}{https://github.com/tuna/thuthesis/compare/v5.4.1...v5.4.2}{https://github.com/tuna/thuthesis/compare/v5.4.1...v5.4.2}{} - 2017-12-18}\markdownRendererInterblockSeparator
{}\markdownRendererHeadingThree{Changed}\markdownRendererInterblockSeparator
{}\markdownRendererUlBegin
\markdownRendererUlItem 删除 \markdownRendererCodeSpan{\markdownRendererBackslash{}pozhehao}。\markdownRendererUlItemEnd 
\markdownRendererUlItem 使用 degree 取代 type 选项。\markdownRendererUlItemEnd 
\markdownRendererUlEnd \markdownRendererInterblockSeparator
{}\markdownRendererHeadingTwo{\markdownRendererLink{v5.4.1}{https://github.com/tuna/thuthesis/compare/v5.4.0...v5.4.1}{https://github.com/tuna/thuthesis/compare/v5.4.0...v5.4.1}{} - 2017-12-04}\markdownRendererInterblockSeparator
{}\markdownRendererHeadingThree{Changed}\markdownRendererInterblockSeparator
{}\markdownRendererUlBegin
\markdownRendererUlItem bst 在 ctan 上不分路径,故加前缀。\markdownRendererUlItemEnd 
\markdownRendererUlEnd \markdownRendererInterblockSeparator
{}\markdownRendererHeadingTwo{\markdownRendererLink{v5.4.0}{https://github.com/tuna/thuthesis/compare/v5.3.2...v5.4.0}{https://github.com/tuna/thuthesis/compare/v5.3.2...v5.4.0}{} - 2017-12-3}\markdownRendererInterblockSeparator
{}\markdownRendererHeadingThree{Changed}\markdownRendererInterblockSeparator
{}\markdownRendererUlBegin
\markdownRendererUlItem 基于 \markdownRendererCodeSpan{natbib} 的环境调整距离兼容性更好。\markdownRendererUlItemEnd 
\markdownRendererUlItem \markdownRendererCodeSpan{\markdownRendererBackslash{}bibliographystyle\markdownRendererLeftBrace{}<newbib>\markdownRendererRightBrace{}} will cause \markdownRendererCodeSpan{\markdownRendererBackslash{}bibstyle@newbib} to be called on THE NEXT LATEX RUN (via the aux file).\markdownRendererUlItemEnd 
\markdownRendererUlEnd \markdownRendererInterblockSeparator
{}\markdownRendererHeadingTwo{\markdownRendererLink{v5.3.2}{https://github.com/tuna/thuthesis/compare/v5.3.1...v5.3.2}{https://github.com/tuna/thuthesis/compare/v5.3.1...v5.3.2}{} - 2017-05-01}\markdownRendererInterblockSeparator
{}\markdownRendererHeadingThree{Changed}\markdownRendererInterblockSeparator
{}\markdownRendererUlBegin
\markdownRendererUlItem 定理环境格式设置(环境标题和环境正文字体设置)统一放置到 .cfg 文件中。\markdownRendererUlItemEnd 
\markdownRendererUlEnd \markdownRendererInterblockSeparator
{}\markdownRendererHeadingTwo{\markdownRendererLink{v5.3.1}{https://github.com/tuna/thuthesis/compare/v5.3.0...v5.3.1}{https://github.com/tuna/thuthesis/compare/v5.3.0...v5.3.1}{} - 2016-03-20}\markdownRendererInterblockSeparator
{}\markdownRendererHeadingThree{Changed}\markdownRendererInterblockSeparator
{}\markdownRendererUlBegin
\markdownRendererUlItem 使用 CTeX 默认中文字体配置,支持不同引擎。\markdownRendererUlItemEnd 
\markdownRendererUlItem \markdownRendererCodeSpan{ctex} 默认加载 \markdownRendererCodeSpan{CJKspace}。\markdownRendererUlItemEnd 
\markdownRendererUlItem 几乎没人主动安装 Arial 字体。\markdownRendererUlItemEnd 
\markdownRendererUlEnd \markdownRendererInterblockSeparator
{}\markdownRendererHeadingTwo{\markdownRendererLink{v5.3.0}{https://github.com/tuna/thuthesis/compare/v5.2.3...v5.3.0}{https://github.com/tuna/thuthesis/compare/v5.2.3...v5.3.0}{} - 2016-03-11}\markdownRendererInterblockSeparator
{}\markdownRendererHeadingThree{Changed}\markdownRendererInterblockSeparator
{}\markdownRendererUlBegin
\markdownRendererUlItem 更新到研究生院 2016.3 指南。\markdownRendererUlItemEnd 
\markdownRendererUlEnd \markdownRendererInterblockSeparator
{}\markdownRendererHeadingTwo{\markdownRendererLink{v5.2.3}{https://github.com/tuna/thuthesis/compare/v5.2.2...v5.2.3}{https://github.com/tuna/thuthesis/compare/v5.2.2...v5.2.3}{} - 2016-02-13}\markdownRendererInterblockSeparator
{}\markdownRendererHeadingThree{Changed}\markdownRendererInterblockSeparator
{}\markdownRendererUlBegin
\markdownRendererUlItem \markdownRendererCodeSpan{\markdownRendererBackslash{}thu@def@fontsize}: 改写字体定义命令。\markdownRendererUlItemEnd 
\markdownRendererUlItem 安全注释本科公式部分。\markdownRendererUlItemEnd 
\markdownRendererUlEnd \markdownRendererInterblockSeparator
{}\markdownRendererHeadingTwo{\markdownRendererLink{v5.2.2}{https://github.com/tuna/thuthesis/compare/v5.2.1...v5.2.2}{https://github.com/tuna/thuthesis/compare/v5.2.1...v5.2.2}{} - 2016-02-01}\markdownRendererInterblockSeparator
{}\markdownRendererHeadingThree{Changed}\markdownRendererInterblockSeparator
{}\markdownRendererUlBegin
\markdownRendererUlItem 不希望 \markdownRendererCodeSpan{newtx} 修改 \markdownRendererCodeSpan{\markdownRendererBackslash{}@makefnmark}。\markdownRendererUlItemEnd 
\markdownRendererUlItem 利用 \markdownRendererCodeSpan{environ} 的 \markdownRendererCodeSpan{\markdownRendererBackslash{}Collect@Body}。\markdownRendererUlItemEnd 
\markdownRendererUlItem 目录中标题和页码都是链接。\markdownRendererUlItemEnd 
\markdownRendererUlItem \markdownRendererCodeSpan{\markdownRendererBackslash{}thu@textcircled}: 脚注编号每页允许至多 9 个。\markdownRendererUlItemEnd 
\markdownRendererUlItem 基于 \markdownRendererCodeSpan{footmisc} 来设置不同位置 footnote marker 样式。\markdownRendererUlItemEnd 
\markdownRendererUlItem \markdownRendererCodeSpan{\markdownRendererBackslash{}tableofcontents}: 用 \markdownRendererCodeSpan{\markdownRendererBackslash{}patchcmd} 修改 \markdownRendererCodeSpan{\markdownRendererBackslash{}@dottedtocline}。\markdownRendererUlItemEnd 
\markdownRendererUlItem 用 \markdownRendererCodeSpan{environ} 封装的 \markdownRendererCodeSpan{\markdownRendererBackslash{}Collect@Body}。\markdownRendererUlItemEnd 
\markdownRendererUlEnd \markdownRendererInterblockSeparator
{}\markdownRendererHeadingTwo{\markdownRendererLink{v5.2.1}{https://github.com/tuna/thuthesis/compare/v5.2.0...v5.2.1}{https://github.com/tuna/thuthesis/compare/v5.2.0...v5.2.1}{} - 2016-01-14}\markdownRendererInterblockSeparator
{}\markdownRendererHeadingThree{Changed}\markdownRendererInterblockSeparator
{}\markdownRendererUlBegin
\markdownRendererUlItem 使用 \markdownRendererCodeSpan{newtx} 替换 \markdownRendererCodeSpan{txfonts}。\markdownRendererUlItemEnd 
\markdownRendererUlItem 使用 \markdownRendererCodeSpan{newtx} 字体。\markdownRendererUlItemEnd 
\markdownRendererUlItem \markdownRendererCodeSpan{denotation}: 利用 \markdownRendererCodeSpan{enumitem} 改造环境定义,更直观。\markdownRendererUlItemEnd 
\markdownRendererUlItem \markdownRendererCodeSpan{acknowledgement}: 用 \markdownRendererCodeSpan{acknowledgement} 替换 \markdownRendererCodeSpan{ack}。\markdownRendererUlItemEnd 
\markdownRendererUlEnd \markdownRendererInterblockSeparator
{}\markdownRendererHeadingTwo{\markdownRendererLink{v5.2.0}{https://github.com/tuna/thuthesis/compare/v5.1.0...v5.2.0}{https://github.com/tuna/thuthesis/compare/v5.1.0...v5.2.0}{} - 2016-01-11}\markdownRendererInterblockSeparator
{}\markdownRendererHeadingThree{Changed}\markdownRendererInterblockSeparator
{}\markdownRendererUlBegin
\markdownRendererUlItem \markdownRendererCodeSpan{\markdownRendererBackslash{}title}: 增加 \markdownRendererCodeSpan{\markdownRendererBackslash{}title} 排版翻译标题。\markdownRendererUlItemEnd 
\markdownRendererUlItem \markdownRendererCodeSpan{translationbib}: 增加翻译文献环境 \markdownRendererCodeSpan{translationbib}。\markdownRendererUlItemEnd 
\markdownRendererUlItem \markdownRendererCodeSpan{\markdownRendererBackslash{}publicationskip}: 增加 \markdownRendererCodeSpan{\markdownRendererBackslash{}publicationskip}。\markdownRendererUlItemEnd 
\markdownRendererUlEnd \markdownRendererInterblockSeparator
{}\markdownRendererHeadingTwo{\markdownRendererLink{v5.1.0}{https://github.com/tuna/thuthesis/compare/v5.0.0...v5.1.0}{https://github.com/tuna/thuthesis/compare/v5.0.0...v5.1.0}{} - 2015-12-27}\markdownRendererInterblockSeparator
{}\markdownRendererHeadingThree{Changed}\markdownRendererInterblockSeparator
{}\markdownRendererUlBegin
\markdownRendererUlItem \markdownRendererCodeSpan{\markdownRendererBackslash{}thusetup}: 通过 \markdownRendererCodeSpan{\markdownRendererBackslash{}thusetup} 统一设置封面信息。\markdownRendererUlItemEnd 
\markdownRendererUlItem \markdownRendererCodeSpan{\markdownRendererBackslash{}thu@first@titlepage}: 利用 \markdownRendererCodeSpan{CJKfilltwosides} 优化封面排版。\markdownRendererUlItemEnd 
\markdownRendererUlItem \markdownRendererCodeSpan{\markdownRendererBackslash{}thu@first@titlepage}: 修改联合指导教师显示问题。\markdownRendererUlItemEnd 
\markdownRendererUlEnd \markdownRendererInterblockSeparator
{}\markdownRendererHeadingTwo{\markdownRendererLink{v5.0.0}{https://github.com/tuna/thuthesis/compare/v4.8.1...v5.0.0}{https://github.com/tuna/thuthesis/compare/v4.8.1...v5.0.0}{} - 2015-12-21}\markdownRendererInterblockSeparator
{}\markdownRendererHeadingThree{Changed}\markdownRendererInterblockSeparator
{}\markdownRendererUlBegin
\markdownRendererUlItem 使用 \markdownRendererCodeSpan{kvoptions} 简化选项 type。\markdownRendererUlItemEnd 
\markdownRendererUlItem norggedbottom 选项修改为 raggedbottom。\markdownRendererUlItemEnd 
\markdownRendererUlItem 删除 \markdownRendererCodeSpan{paralist} 选项。\markdownRendererUlItemEnd 
\markdownRendererUlItem 固定字体设置,同时改善与 \markdownRendererCodeSpan{ctex} 兼容性。\markdownRendererUlItemEnd 
\markdownRendererUlItem 不再将页面尺寸写入 dvi,因为已不支持 dvips, 而该方案会使得在使用 tikzexternalize 时外部 PDF 图片 BBox 不对。\markdownRendererUlItemEnd 
\markdownRendererUlItem 用 \markdownRendererCodeSpan{geometry} 简化设置。\markdownRendererUlItemEnd 
\markdownRendererUlItem \markdownRendererCodeSpan{\markdownRendererBackslash{}ps@thu@headings}: 利用 \markdownRendererCodeSpan{fancyhdr} 设置页眉页脚。\markdownRendererUlItemEnd 
\markdownRendererUlItem 修正定理字样为黑体(\markdownRendererLink{\markdownRendererHash{}104}{https://github.com/tuna/thuthesis/issues/104}{https://github.com/tuna/thuthesis/issues/104}{})。\markdownRendererUlItemEnd 
\markdownRendererUlItem 本科附录图表编号用-不用.(如图A-1,表A-2)。\markdownRendererUlItemEnd 
\markdownRendererUlItem 用 \markdownRendererCodeSpan{\markdownRendererBackslash{}ctexset} 来设置,替换复杂的 \markdownRendererCodeSpan{\markdownRendererBackslash{}@startsection}。\markdownRendererUlItemEnd 
\markdownRendererUlItem 修正章节间距问题(\markdownRendererLink{\markdownRendererHash{}57}{https://github.com/tuna/thuthesis/issues/57}{https://github.com/tuna/thuthesis/issues/57}{})。\markdownRendererUlItemEnd 
\markdownRendererUlItem 硕士博士论文目录只出现到第 3 级标题即可。其他未明确要求。\markdownRendererUlItemEnd 
\markdownRendererUlItem \markdownRendererCodeSpan{\markdownRendererBackslash{}tableofcontents}: 修正学位论文中目录里节前缩进(\markdownRendererLink{\markdownRendererHash{}103}{https://github.com/tuna/thuthesis/issues/103}{https://github.com/tuna/thuthesis/issues/103}{})。\markdownRendererUlItemEnd 
\markdownRendererUlItem \markdownRendererCodeSpan{\markdownRendererBackslash{}makecover}: 使用 \markdownRendererCodeSpan{pdfpages} 宏包支持本硕博论文授权说明扫描版(\markdownRendererLink{\markdownRendererHash{}36}{https://github.com/tuna/thuthesis/issues/36}{https://github.com/tuna/thuthesis/issues/36}{})。\markdownRendererUlItemEnd 
\markdownRendererUlItem \markdownRendererCodeSpan{acknowledgement}: 使用 pdfpages 宏包支持本硕博论文声明扫描版(\markdownRendererLink{\markdownRendererHash{}36}{https://github.com/tuna/thuthesis/issues/36}{https://github.com/tuna/thuthesis/issues/36}{})。\markdownRendererUlItemEnd 
\markdownRendererUlItem \markdownRendererCodeSpan{\markdownRendererBackslash{}inlinecite}: 用 \markdownRendererCodeSpan{\markdownRendererBackslash{}inlinecite} 替换 \markdownRendererCodeSpan{\markdownRendererBackslash{}onlinecite}。为保证兼 容性,\markdownRendererCodeSpan{\markdownRendererBackslash{}onlinecite} 会保留。\markdownRendererUlItemEnd 
\markdownRendererUlItem \markdownRendererCodeSpan{achievements}: 博士后就不提在学期间了,不合适(\markdownRendererLink{\markdownRendererHash{}100}{https://github.com/tuna/thuthesis/issues/100}{https://github.com/tuna/thuthesis/issues/100}{})。\markdownRendererUlItemEnd 
\markdownRendererUlItem \markdownRendererCodeSpan{achievements}: 让简历部分更符合格式指南和示例文件(\markdownRendererLink{\markdownRendererHash{}122}{https://github.com/tuna/thuthesis/issues/122}{https://github.com/tuna/thuthesis/issues/122}{})。\markdownRendererUlItemEnd 
\markdownRendererUlItem \markdownRendererCodeSpan{\markdownRendererBackslash{}shuji}: 扩展 \markdownRendererCodeSpan{\markdownRendererBackslash{}shuji[<标题>][<作者>]}。\markdownRendererUlItemEnd 
\markdownRendererUlEnd \markdownRendererInterblockSeparator
{}\markdownRendererHeadingTwo{\markdownRendererLink{v4.8.1}{https://github.com/tuna/thuthesis/compare/v4.8...v4.8.1}{https://github.com/tuna/thuthesis/compare/v4.8...v4.8.1}{} - 2014-12-09}\markdownRendererInterblockSeparator
{}\markdownRendererHeadingThree{Changed}\markdownRendererInterblockSeparator
{}\markdownRendererUlBegin
\markdownRendererUlItem 按照 CTAN 的要求整理一下文件。\markdownRendererUlItemEnd 
\markdownRendererUlEnd \markdownRendererInterblockSeparator
{}\markdownRendererHeadingTwo{\markdownRendererLink{v4.8}{https://github.com/tuna/thuthesis/compare/v4.7...v4.8}{https://github.com/tuna/thuthesis/compare/v4.7...v4.8}{} - 2014-11-25}\markdownRendererInterblockSeparator
{}\markdownRendererHeadingThree{Changed}\markdownRendererInterblockSeparator
{}\markdownRendererUlBegin
\markdownRendererUlItem no need to load \markdownRendererCodeSpan{indentfirst} directly since we use \markdownRendererCodeSpan{ctex}.\markdownRendererUlItemEnd 
\markdownRendererUlItem 内部调用 \markdownRendererCodeSpan{ctex} 宏包,自动检测编译引擎。\markdownRendererUlItemEnd 
\markdownRendererUlItem dvips method is deprecated. We ask their users to load it manually.\markdownRendererUlItemEnd 
\markdownRendererUlItem reset baselinestretch after ctex's change.\markdownRendererUlItemEnd 
\markdownRendererUlItem 好几年累积的一些更新,最重要的是切换到 CTeX。\markdownRendererUlItemEnd 
\markdownRendererUlItem v4.7曾经想发布,但是一直没有做,于是就被跳过了,算是造一个段子吧。\markdownRendererUlItemEnd 
\markdownRendererUlItem 增加 noraggedbottom 选项。\markdownRendererUlItemEnd 
\markdownRendererUlItem 添加 nocap 选项,恢复默认标题样式,模板会进一步定制。\markdownRendererUlItemEnd 
\markdownRendererUlItem no need to load amssymb since we use txfonts.\markdownRendererUlItemEnd 
\markdownRendererUlItem 在 CJK 模式下用 \markdownRendererCodeSpan{CJKspace} 保留中英文间空格。\markdownRendererUlItemEnd 
\markdownRendererUlEnd \markdownRendererInterblockSeparator
{}\markdownRendererHeadingTwo{\markdownRendererLink{v4.7}{https://github.com/tuna/thuthesis/compare/v4.6...v4.7}{https://github.com/tuna/thuthesis/compare/v4.6...v4.7}{} - 2012-06-12}\markdownRendererInterblockSeparator
{}\markdownRendererHeadingThree{Changed}\markdownRendererInterblockSeparator
{}\markdownRendererUlBegin
\markdownRendererUlItem 去掉 \markdownRendererCodeSpan{hypernat} 依赖,\markdownRendererCodeSpan{hyperref} 和 \markdownRendererCodeSpan{natbib} 可以很好配合了。\markdownRendererUlItemEnd 
\markdownRendererUlItem 修改本科生页脚间距与样例基本一致。\markdownRendererUlItemEnd 
\markdownRendererUlItem \markdownRendererCodeSpan{\markdownRendererBackslash{}ps@thu@headings}: 本科页码用小五号字。\markdownRendererUlItemEnd 
\markdownRendererUlItem 修正本科生作者信息名称。\markdownRendererUlItemEnd 
\markdownRendererUlItem 本科生关键字也用分号分割了。\markdownRendererUlItemEnd 
\markdownRendererUlItem \markdownRendererCodeSpan{\markdownRendererBackslash{}thu@first@titlepage}: 硕士中文封面不再需要英文标题。\markdownRendererUlItemEnd 
\markdownRendererUlItem \markdownRendererCodeSpan{\markdownRendererBackslash{}thu@first@titlepage}: 本科生题目下划线长度自动适应字数。\markdownRendererUlItemEnd 
\markdownRendererUlItem \markdownRendererCodeSpan{\markdownRendererBackslash{}thu@doctor@engcover}: 硕士生新增英文封面。\markdownRendererUlItemEnd 
\markdownRendererUlItem \markdownRendererCodeSpan{\markdownRendererBackslash{}makecover}: 硕士论文也需要英文封面。\markdownRendererUlItemEnd 
\markdownRendererUlItem \markdownRendererCodeSpan{\markdownRendererBackslash{}thu@makeabstract}: Bachelor sample uses Keywords w/o space \markdownRendererCodeSpan{-\markdownRendererUnderscore{}-}\markdownRendererUlItemEnd 
\markdownRendererUlEnd \markdownRendererInterblockSeparator
{}\markdownRendererHeadingTwo{\markdownRendererLink{v4.6}{https://github.com/tuna/thuthesis/compare/v4.5.2...v4.6}{https://github.com/tuna/thuthesis/compare/v4.5.2...v4.6}{} - 2011-10-22}\markdownRendererInterblockSeparator
{}\markdownRendererHeadingThree{Changed}\markdownRendererInterblockSeparator
{}\markdownRendererUlBegin
\markdownRendererUlItem 增加博士后文档部分。\markdownRendererUlItemEnd 
\markdownRendererUlItem 使用手册更新。\markdownRendererUlItemEnd 
\markdownRendererUlItem 增加 postdoctor 选项。\markdownRendererUlItemEnd 
\markdownRendererUlItem 增加博士后相关指令。\markdownRendererUlItemEnd 
\markdownRendererUlItem 增加博士后相关配置。\markdownRendererUlItemEnd 
\markdownRendererUlItem \markdownRendererCodeSpan{\markdownRendererBackslash{}thu@first@titlepage}: 增加博士后封面。\markdownRendererUlItemEnd 
\markdownRendererUlItem \markdownRendererCodeSpan{\markdownRendererBackslash{}makecover}: 博士后报告无授权说明。\markdownRendererUlItemEnd 
\markdownRendererUlItem \markdownRendererCodeSpan{resume}: 支持可选参数,自己定义简历章节标题。\markdownRendererUlItemEnd 
\markdownRendererUlEnd \markdownRendererInterblockSeparator
{}\markdownRendererHeadingTwo{\markdownRendererLink{v4.5.2}{https://github.com/tuna/thuthesis/compare/v4.5.1...v4.5.2}{https://github.com/tuna/thuthesis/compare/v4.5.1...v4.5.2}{} - 2010-09-19}\markdownRendererInterblockSeparator
{}\markdownRendererHeadingThree{Changed}\markdownRendererInterblockSeparator
{}\markdownRendererUlBegin
\markdownRendererUlItem 研究生页面边距由 3.2cm 改为 3cm。\markdownRendererUlItemEnd 
\markdownRendererUlItem 本科论文日期具体到日。\markdownRendererUlItemEnd 
\markdownRendererUlItem \markdownRendererCodeSpan{\markdownRendererBackslash{}makecover}: 本科封面和授权说明之间不要空白页。\markdownRendererUlItemEnd 
\markdownRendererUlItem \markdownRendererCodeSpan{\markdownRendererBackslash{}thu@makeabstract}: 本科论文摘要亦无需右开。\markdownRendererUlItemEnd 
\markdownRendererUlItem \markdownRendererCodeSpan{acknowledgement}: 研究生论文的致谢和声明终于分开了。\markdownRendererUlItemEnd 
\markdownRendererUlEnd \markdownRendererInterblockSeparator
{}\markdownRendererHeadingTwo{\markdownRendererLink{v4.5.1}{https://github.com/tuna/thuthesis/compare/v4.5...v4.5.1}{https://github.com/tuna/thuthesis/compare/v4.5...v4.5.1}{} - 2009-01-06}\markdownRendererInterblockSeparator
{}\markdownRendererHeadingThree{Changed}\markdownRendererInterblockSeparator
{}\markdownRendererUlBegin
\markdownRendererUlItem 太好了,不用处理 \markdownRendererCodeSpan{longtable} 的 \markdownRendererCodeSpan{\markdownRendererBackslash{}caption} 了。\markdownRendererUlItemEnd 
\markdownRendererUlItem \markdownRendererCodeSpan{\markdownRendererBackslash{}listoftables*}: 更优雅的插图/表格索引,避免跟 \markdownRendererCodeSpan{caption} 包冲 突。\markdownRendererCodeSpan{\markdownRendererBackslash{}thu@listof} 相应修改。\markdownRendererUlItemEnd 
\markdownRendererUlEnd \markdownRendererInterblockSeparator
{}\markdownRendererHeadingTwo{\markdownRendererLink{v4.5}{https://github.com/tuna/thuthesis/compare/v4.4.4...v4.5}{https://github.com/tuna/thuthesis/compare/v4.4.4...v4.5}{} - 2009-01-04}\markdownRendererInterblockSeparator
{}\markdownRendererHeadingThree{Changed}\markdownRendererInterblockSeparator
{}\markdownRendererUlBegin
\markdownRendererUlItem 加入 XeTeX 支持,需要 \markdownRendererCodeSpan{xeCJK}。\markdownRendererUlItemEnd 
\markdownRendererUlItem 彻底转向 UTF-8,并支持 XeLaTeX。\markdownRendererUlItemEnd 
\markdownRendererUlItem 增加 xetex, pdftex 选项。\markdownRendererUlItemEnd 
\markdownRendererUlItem \markdownRendererCodeSpan{\markdownRendererBackslash{}shuji}: 简化代码,同时支持 XeLaTeX。\markdownRendererUlItemEnd 
\markdownRendererUlEnd \markdownRendererInterblockSeparator
{}\markdownRendererHeadingTwo{\markdownRendererLink{v4.4.4}{https://github.com/tuna/thuthesis/compare/v4.4.3...v4.4.4}{https://github.com/tuna/thuthesis/compare/v4.4.3...v4.4.4}{} - 2008-06-12}\markdownRendererInterblockSeparator
{}\markdownRendererHeadingThree{Changed}\markdownRendererInterblockSeparator
{}\markdownRendererUlBegin
\markdownRendererUlItem 修复了一个从 v4.3 升级到 v4.4 过程中的丢失公式索引的 bug,原修改代码保留备忘。\markdownRendererUlItemEnd 
\markdownRendererUlEnd \markdownRendererInterblockSeparator
{}\markdownRendererHeadingTwo{\markdownRendererLink{v4.4.3}{https://github.com/tuna/thuthesis/compare/v4.4.2...v4.4.3}{https://github.com/tuna/thuthesis/compare/v4.4.2...v4.4.3}{} - 2008-06-09}\markdownRendererInterblockSeparator
{}\markdownRendererHeadingThree{Changed}\markdownRendererInterblockSeparator
{}\markdownRendererUlBegin
\markdownRendererUlItem \markdownRendererCodeSpan{\markdownRendererBackslash{}thu@first@titlepage}: 修改本科生论文封面格式以符合新样例。\markdownRendererUlItemEnd 
\markdownRendererUlItem \markdownRendererCodeSpan{\markdownRendererBackslash{}thu@first@titlepage}: 修改本科生论文封面日期格式以符合新样例。\markdownRendererUlItemEnd 
\markdownRendererUlEnd \markdownRendererInterblockSeparator
{}\markdownRendererHeadingTwo{\markdownRendererLink{v4.4.2}{https://github.com/tuna/thuthesis/compare/v4.4...v4.4.2}{https://github.com/tuna/thuthesis/compare/v4.4...v4.4.2}{} - 2008-06-07}\markdownRendererInterblockSeparator
{}\markdownRendererHeadingThree{Changed}\markdownRendererInterblockSeparator
{}\markdownRendererUlBegin
\markdownRendererUlItem 本科生格式终于也开始用空格作为关键字分隔符了。\markdownRendererUlItemEnd 
\markdownRendererUlItem 本科生签名之间距离改为 \markdownRendererCodeSpan{\markdownRendererBackslash{}hskip1em}。\markdownRendererUlItemEnd 
\markdownRendererUlItem \markdownRendererCodeSpan{\markdownRendererBackslash{}thu@authorization@mk}: 修改本科生的授权部分,按照 2008 年的新样例。\markdownRendererUlItemEnd 
\markdownRendererUlItem \markdownRendererCodeSpan{\markdownRendererBackslash{}thu@makeabstract}: 本科生格式中文关键词采用首行缩进且无悬挂缩进。\markdownRendererUlItemEnd 
\markdownRendererUlItem \markdownRendererCodeSpan{\markdownRendererBackslash{}thu@makeabstract}: Bachelor English abstract format requires indent and no hang-indent.\markdownRendererUlItemEnd 
\markdownRendererUlEnd \markdownRendererInterblockSeparator
{}\markdownRendererHeadingTwo{\markdownRendererLink{v4.4}{https://github.com/tuna/thuthesis/compare/v4.3...v4.4}{https://github.com/tuna/thuthesis/compare/v4.3...v4.4}{} - 2008-06-18}\markdownRendererInterblockSeparator
{}\markdownRendererHeadingThree{Changed}\markdownRendererInterblockSeparator
{}\markdownRendererUlBegin
\markdownRendererUlItem 修复网址断字。\markdownRendererUlItemEnd 
\markdownRendererUlItem \markdownRendererCodeSpan{\markdownRendererBackslash{}backmatter}: 本科正文后的页码延续前面的阿拉伯数字,不再用罗马数 字。\markdownRendererUlItemEnd 
\markdownRendererUlItem \markdownRendererCodeSpan{\markdownRendererBackslash{}backmatter}: 本科取消了所有页眉。\markdownRendererUlItemEnd 
\markdownRendererUlItem 本科论文终于去掉了\markdownRendererStrongEmphasis{公式}二字。\markdownRendererUlItemEnd 
\markdownRendererUlItem 调整段前距为 -20bp 而不是原来的 -24bp。\markdownRendererUlItemEnd 
\markdownRendererUlItem 修改本科生模板的二级节标题为小四而不是半小四。\markdownRendererUlItemEnd 
\markdownRendererUlItem 调整段前距为 -12bp 而不是原来的 -16bp。\markdownRendererUlItemEnd 
\markdownRendererUlItem 调整段前距为 -12bp 而不是原来的 -16bp。\markdownRendererUlItemEnd 
\markdownRendererUlItem \markdownRendererCodeSpan{\markdownRendererBackslash{}tableofcontents}: 本科生目录字号改回\markdownRendererCodeSpan{\markdownRendererBackslash{}xiaosi[<1.8>]}。\markdownRendererUlItemEnd 
\markdownRendererUlItem \markdownRendererCodeSpan{\markdownRendererBackslash{}tableofcontents}: 本科生目录缩进要求不同。\markdownRendererUlItemEnd 
\markdownRendererUlItem \markdownRendererCodeSpan{\markdownRendererBackslash{}tableofcontents}: 本科章目录项一直用黑体(Arial)。\markdownRendererUlItemEnd 
\markdownRendererUlEnd \markdownRendererInterblockSeparator
{}\markdownRendererHeadingTwo{\markdownRendererLink{v4.3}{https://github.com/tuna/thuthesis/compare/v4.2...v4.3}{https://github.com/tuna/thuthesis/compare/v4.2...v4.3}{} - 2008-03-11}\markdownRendererInterblockSeparator
{}\markdownRendererHeadingThree{Changed}\markdownRendererInterblockSeparator
{}\markdownRendererUlBegin
\markdownRendererUlItem 子图引用时加括号。\markdownRendererUlItemEnd 
\markdownRendererUlEnd \markdownRendererInterblockSeparator
{}\markdownRendererHeadingTwo{\markdownRendererLink{v4.2}{https://github.com/tuna/thuthesis/compare/v4.0...v4.2}{https://github.com/tuna/thuthesis/compare/v4.0...v4.2}{} - 2008-03-07}\markdownRendererInterblockSeparator
{}\markdownRendererHeadingThree{Changed}\markdownRendererInterblockSeparator
{}\markdownRendererUlBegin
\markdownRendererUlItem \markdownRendererCodeSpan{\markdownRendererBackslash{}eqref} 加括号。\markdownRendererUlItemEnd 
\markdownRendererUlItem 调整证明环境的编号和结尾的方块。\markdownRendererUlItemEnd 
\markdownRendererUlItem \markdownRendererCodeSpan{\markdownRendererBackslash{}thu@doctor@engcover}: 博士英文封面补充联合导师。\markdownRendererUlItemEnd 
\markdownRendererUlEnd \markdownRendererInterblockSeparator
{}\markdownRendererHeadingTwo{\markdownRendererLink{v4.0}{https://github.com/tuna/thuthesis/compare/v3.1...v4.0}{https://github.com/tuna/thuthesis/compare/v3.1...v4.0}{} - 2007-11-08}\markdownRendererInterblockSeparator
{}\markdownRendererHeadingThree{Changed}\markdownRendererInterblockSeparator
{}\markdownRendererUlBegin
\markdownRendererUlItem \markdownRendererCodeSpan{\markdownRendererBackslash{}tableofcontents}: 本科研究生目录字号行距都不同。\markdownRendererUlItemEnd 
\markdownRendererUlItem \markdownRendererStrongEmphasis{内部}密级前面终究还是不要五角星了。\markdownRendererUlItemEnd 
\markdownRendererUlItem \markdownRendererCodeSpan{\markdownRendererBackslash{}thu@authorization@mk}: 研究生的授权部分调整了一下,不知道老师为什么总爱修改 那些无关紧要的格式,郁闷。感谢 PMHT@newsmth 的认真比对。\markdownRendererUlItemEnd 
\markdownRendererUlEnd \markdownRendererInterblockSeparator
{}\markdownRendererHeadingTwo{\markdownRendererLink{v3.1}{https://github.com/tuna/thuthesis/compare/v3.0...v3.1}{https://github.com/tuna/thuthesis/compare/v3.0...v3.1}{} - 2007-10-09}\markdownRendererInterblockSeparator
{}\markdownRendererHeadingThree{Changed}\markdownRendererInterblockSeparator
{}\markdownRendererUlBegin
\markdownRendererUlItem 本科的目录又不要 arial 字体了。\markdownRendererUlItemEnd 
\markdownRendererUlItem replace \markdownRendererCodeSpan{mathptmx} with \markdownRendererCodeSpan{txfonts}.\markdownRendererUlItemEnd 
\markdownRendererUlItem 英文摘要标题要搞特殊化。\markdownRendererUlItemEnd 
\markdownRendererUlItem 博士论文目录只出现到第 3 级标题即可。\markdownRendererUlItemEnd 
\markdownRendererUlItem \markdownRendererCodeSpan{\markdownRendererBackslash{}thu@def@term}: 重新定义摘要为环境,long 选项不需要了。\markdownRendererUlItemEnd 
\markdownRendererUlItem 重新定义摘要成为环境。\markdownRendererUlItemEnd 
\markdownRendererUlItem 增强的关键词命令。\markdownRendererUlItemEnd 
\markdownRendererUlItem 去掉配置文件中的 \markdownRendererCodeSpan{\markdownRendererBackslash{}hfill}。\markdownRendererUlItemEnd 
\markdownRendererUlItem \markdownRendererStrongEmphasis{内部}密级前面要五角星了。\markdownRendererUlItemEnd 
\markdownRendererUlItem \markdownRendererCodeSpan{\markdownRendererBackslash{}thu@first@titlepage}: 重新放置封面表格的提示元素。\markdownRendererUlItemEnd 
\markdownRendererUlItem \markdownRendererCodeSpan{\markdownRendererBackslash{}thu@makeabstract}: 研究生关键词不再沉底。\markdownRendererUlItemEnd 
\markdownRendererUlEnd \markdownRendererInterblockSeparator
{}\markdownRendererHeadingTwo{\markdownRendererLink{v3.0}{https://github.com/tuna/thuthesis/compare/v2.6.4...v3.0}{https://github.com/tuna/thuthesis/compare/v2.6.4...v3.0}{} - 2007-05-13}\markdownRendererInterblockSeparator
{}\markdownRendererHeadingThree{Changed}\markdownRendererInterblockSeparator
{}\markdownRendererUlBegin
\markdownRendererUlItem 不用专门为本科论文生成“\markdownRendererStrongEmphasis{提交}”版本了。\markdownRendererUlItemEnd 
\markdownRendererUlItem 没有了综合论文训练页面,很多本科论文专用命令就消失了。\markdownRendererUlItemEnd 
\markdownRendererUlItem 删除 submit 选项。\markdownRendererUlItemEnd 
\markdownRendererUlItem 本科公式又要取消全文统一编号了。\markdownRendererUlItemEnd 
\markdownRendererUlItem \markdownRendererCodeSpan{\markdownRendererBackslash{}tableofcontents}: 缩小目录中标题与页码之间\markdownRendererStrongEmphasis{点}之间的距离。\markdownRendererUlItemEnd 
\markdownRendererUlItem \markdownRendererCodeSpan{\markdownRendererBackslash{}makecover}: 本科论文评语取消。\markdownRendererUlItemEnd 
\markdownRendererUlItem \markdownRendererCodeSpan{\markdownRendererBackslash{}makecover}: 本科论文授权图片扫描取消。\markdownRendererUlItemEnd 
\markdownRendererUlItem \markdownRendererCodeSpan{\markdownRendererBackslash{}makecover}: 本科综合论文训练在电子版中取消。\markdownRendererUlItemEnd 
\markdownRendererUlItem \markdownRendererCodeSpan{\markdownRendererBackslash{}thu@makeabstract}: \markdownRendererStrongEmphasis{Key words} but not \markdownRendererStrongEmphasis{Keywords}. What are you doing?\markdownRendererUlItemEnd 
\markdownRendererUlItem \markdownRendererCodeSpan{acknowledgement}: 本科论文声明部分图片扫描取消。\markdownRendererUlItemEnd 
\markdownRendererUlEnd \markdownRendererInterblockSeparator
{}\markdownRendererHeadingTwo{\markdownRendererLink{v2.6.4}{https://github.com/tuna/thuthesis/compare/v2.6.3...v2.6.4}{https://github.com/tuna/thuthesis/compare/v2.6.3...v2.6.4}{} - 2006-10-23}\markdownRendererInterblockSeparator
{}\markdownRendererHeadingThree{Changed}\markdownRendererInterblockSeparator
{}\markdownRendererUlBegin
\markdownRendererUlItem 增加 \markdownRendererCodeSpan{neverdecrease} 选项。\markdownRendererUlItemEnd 
\markdownRendererUlItem \markdownRendererCodeSpan{\markdownRendererBackslash{}thu@makeabstract}: \markdownRendererStrongEmphasis{Keywords} but not \markdownRendererStrongEmphasis{Key words}.\markdownRendererUlItemEnd 
\markdownRendererUlItem \markdownRendererCodeSpan{\markdownRendererBackslash{}listoftables*}: 增加 \markdownRendererCodeSpan{\markdownRendererBackslash{}listoffigures*},\markdownRendererCodeSpan{\markdownRendererBackslash{}listoftables*}。\markdownRendererUlItemEnd 
\markdownRendererUlItem \markdownRendererCodeSpan{\markdownRendererBackslash{}listofequations*}: 增加 \markdownRendererCodeSpan{\markdownRendererBackslash{}listofequations*}。\markdownRendererUlItemEnd 
\markdownRendererUlItem 调整参考文献标签宽度,使得条目增多时仍能对齐。\markdownRendererUlItemEnd 
\markdownRendererUlEnd \markdownRendererInterblockSeparator
{}\markdownRendererHeadingTwo{\markdownRendererLink{v2.6.3}{https://github.com/tuna/thuthesis/compare/v2.6.2...v2.6.3}{https://github.com/tuna/thuthesis/compare/v2.6.2...v2.6.3}{} - 2006-07-01}\markdownRendererInterblockSeparator
{}\markdownRendererHeadingThree{Changed}\markdownRendererInterblockSeparator
{}\markdownRendererUlBegin
\markdownRendererUlItem \markdownRendererCodeSpan{\markdownRendererBackslash{}thu@makeabstract}: 为本科正确设置目录及以后的页码。\markdownRendererUlItemEnd 
\markdownRendererUlItem \markdownRendererCodeSpan{acknowledgement}: 重画双虚线,自适应页面宽度。\markdownRendererUlItemEnd 
\markdownRendererUlEnd \markdownRendererInterblockSeparator
{}\markdownRendererHeadingTwo{\markdownRendererLink{v2.6.2}{https://github.com/tuna/thuthesis/compare/v2.6.1...v2.6.2}{https://github.com/tuna/thuthesis/compare/v2.6.1...v2.6.2}{} - 2006-06-20}\markdownRendererInterblockSeparator
{}\markdownRendererHeadingThree{Changed}\markdownRendererInterblockSeparator
{}\markdownRendererUlBegin
\markdownRendererUlItem 改正 groupmembers 的拼写错误。\markdownRendererUlItemEnd 
\markdownRendererUlItem 去掉 \markdownRendererCodeSpan{paralist} 的 \markdownRendererCodeSpan{newitem} 和 \markdownRendererCodeSpan{newenum} 选项,因为默认是打开的。\markdownRendererUlItemEnd 
\markdownRendererUlItem \markdownRendererCodeSpan{\markdownRendererBackslash{}thu@def@fontsize}: 引入此命令重新定义字号。\markdownRendererUlItemEnd 
\markdownRendererUlItem 根据不同论文格式显示不同公式编号,并自动加入索引。\markdownRendererUlItemEnd 
\markdownRendererUlItem 增加问题和猜想两个数学环境。\markdownRendererUlItemEnd 
\markdownRendererUlItem \markdownRendererCodeSpan{\markdownRendererBackslash{}thu@def@term}: 引入 \markdownRendererCodeSpan{\markdownRendererBackslash{}thu@def@term} 定义封面命令。\markdownRendererUlItemEnd 
\markdownRendererUlItem \markdownRendererCodeSpan{\markdownRendererBackslash{}thu@first@titlepage}: 如果本科生没有辅导教师则不显示。\markdownRendererUlItemEnd 
\markdownRendererUlItem \markdownRendererCodeSpan{\markdownRendererBackslash{}thu@makeabstract}: 取消最后一列的空白。\markdownRendererUlItemEnd 
\markdownRendererUlItem \markdownRendererCodeSpan{\markdownRendererBackslash{}thu@makeabstract}: 取消 tabular 环境,用 \markdownRendererCodeSpan{\markdownRendererBackslash{}hangindent} 实现关键词 悬挂缩进,英文摘要同。\markdownRendererUlItemEnd 
\markdownRendererUlItem \markdownRendererCodeSpan{\markdownRendererBackslash{}thu@makeabstract}: 取消最后一列的空白。\markdownRendererUlItemEnd 
\markdownRendererUlItem \markdownRendererCodeSpan{\markdownRendererBackslash{}equcaption}: 此命令配合 \markdownRendererCodeSpan{amsmath} 命令基本可以满足所有 公式需要。\markdownRendererUlItemEnd 
\markdownRendererUlEnd \markdownRendererInterblockSeparator
{}\markdownRendererHeadingTwo{\markdownRendererLink{v2.6.1}{https://github.com/tuna/thuthesis/compare/v2.6...v2.6.1}{https://github.com/tuna/thuthesis/compare/v2.6...v2.6.1}{} - 2006-06-16}\markdownRendererInterblockSeparator
{}\markdownRendererHeadingThree{Changed}\markdownRendererInterblockSeparator
{}\markdownRendererUlBegin
\markdownRendererUlItem 取消 \markdownRendererCodeSpan{thubib.bst} 中 inbook 类 volume 后的页 码。\markdownRendererUlItemEnd 
\markdownRendererUlEnd \markdownRendererInterblockSeparator
{}\markdownRendererHeadingTwo{\markdownRendererLink{v2.6}{https://github.com/tuna/thuthesis/compare/v2.5.3...v2.6}{https://github.com/tuna/thuthesis/compare/v2.5.3...v2.6}{} - 2006-06-09}\markdownRendererInterblockSeparator
{}\markdownRendererHeadingThree{Changed}\markdownRendererInterblockSeparator
{}\markdownRendererUlBegin
\markdownRendererUlItem 增加 dvipdfm 选项。\markdownRendererUlItemEnd 
\markdownRendererUlItem 增加 \markdownRendererCodeSpan{longtable}。\markdownRendererUlItemEnd 
\markdownRendererUlItem 去除 hyperref 选项,等待全局传递。\markdownRendererUlItemEnd 
\markdownRendererUlItem 脚注改成 1.5 倍行距,漂亮。\markdownRendererUlItemEnd 
\markdownRendererUlItem 增加 \markdownRendererCodeSpan{\markdownRendererBackslash{}floatsep},\markdownRendererCodeSpan{\markdownRendererBackslash{}@fptop},\markdownRendererCodeSpan{\markdownRendererBackslash{}@fpsep} 和 \markdownRendererCodeSpan{\markdownRendererBackslash{}@fpbot}。\markdownRendererUlItemEnd 
\markdownRendererUlItem \markdownRendererCodeSpan{\markdownRendererBackslash{}thu@first@titlepage}: 本科生题目加长,最多 24 个字。\markdownRendererUlItemEnd 
\markdownRendererUlEnd \markdownRendererInterblockSeparator
{}\markdownRendererHeadingTwo{\markdownRendererLink{v2.5.3}{https://github.com/tuna/thuthesis/compare/v2.5.2...v2.5.3}{https://github.com/tuna/thuthesis/compare/v2.5.2...v2.5.3}{} - 2006-06-08}\markdownRendererInterblockSeparator
{}\markdownRendererHeadingThree{Changed}\markdownRendererInterblockSeparator
{}\markdownRendererUlBegin
\markdownRendererUlItem submit 选项的一个笔误。\markdownRendererUlItemEnd 
\markdownRendererUlItem \markdownRendererCodeSpan{\markdownRendererBackslash{}backmatter}: 第一章永远右开。\markdownRendererUlItemEnd 
\markdownRendererUlItem 不管 caption 出现在什么位置,\markdownRendererCodeSpan{\markdownRendererBackslash{}aboveskip} 总是出现在标题和浮动体之间的距离。\markdownRendererUlItemEnd 
\markdownRendererUlItem 增加对 \markdownRendererCodeSpan{longtable} 的处理。\markdownRendererUlItemEnd 
\markdownRendererUlItem \markdownRendererCodeSpan{\markdownRendererBackslash{}thu@makeabstract}: \markdownRendererCodeSpan{\markdownRendererBackslash{}pagenumber} 会自动设置页码为 1。\markdownRendererUlItemEnd 
\markdownRendererUlItem \markdownRendererCodeSpan{\markdownRendererBackslash{}equcaption}: 取消 \markdownRendererCodeSpan{\markdownRendererBackslash{}equcaption} 的参数\markdownRendererUlItemEnd 
\markdownRendererUlEnd \markdownRendererInterblockSeparator
{}\markdownRendererHeadingTwo{\markdownRendererLink{v2.5.2}{https://github.com/tuna/thuthesis/compare/v2.5.1...v2.5.2}{https://github.com/tuna/thuthesis/compare/v2.5.1...v2.5.2}{} - 2006-06-01}\markdownRendererInterblockSeparator
{}\markdownRendererHeadingThree{Changed}\markdownRendererInterblockSeparator
{}\markdownRendererUlBegin
\markdownRendererUlItem 更改默认列表距离。\markdownRendererUlItemEnd 
\markdownRendererUlItem 上一个版本忘了把研究生的公式编号排除。\markdownRendererUlItemEnd 
\markdownRendererUlItem \markdownRendererCodeSpan{\markdownRendererBackslash{}thu@chapter*}: 定义自己的 \markdownRendererCodeSpan{\markdownRendererBackslash{}thu@chapter*}。\markdownRendererUlItemEnd 
\markdownRendererUlItem \markdownRendererCodeSpan{\markdownRendererBackslash{}tableofcontents}: 用 \markdownRendererCodeSpan{\markdownRendererBackslash{}thu@chapter*} 改写目录命令。\markdownRendererUlItemEnd 
\markdownRendererUlItem \markdownRendererCodeSpan{\markdownRendererBackslash{}thu@first@titlepage}: 研究生论文标题中英文用 arial 字体。\markdownRendererUlItemEnd 
\markdownRendererUlItem \markdownRendererCodeSpan{\markdownRendererBackslash{}thu@makeabstract}: 在研究生论文中,摘要不出现在目录中,但是要在书签中出现。\markdownRendererUlItemEnd 
\markdownRendererUlItem \markdownRendererCodeSpan{acknowledgement}: 研究生致谢右开。\markdownRendererUlItemEnd 
\markdownRendererUlItem \markdownRendererCodeSpan{acknowledgement}: 研究生致谢题目是致谢,目录是致谢与声明。\markdownRendererUlItemEnd 
\markdownRendererUlItem \markdownRendererCodeSpan{resume}: 研究生的个人介绍要右开。\markdownRendererUlItemEnd 
\markdownRendererUlEnd \markdownRendererInterblockSeparator
{}\markdownRendererHeadingTwo{\markdownRendererLink{v2.5.1}{https://github.com/tuna/thuthesis/compare/v2.5...v2.5.1}{https://github.com/tuna/thuthesis/compare/v2.5...v2.5.1}{} - 2006-05-28}\markdownRendererInterblockSeparator
{}\markdownRendererHeadingThree{Changed}\markdownRendererInterblockSeparator
{}\markdownRendererUlBegin
\markdownRendererUlItem 如果选项设置了 dvips,但是用 PDFLaTeX 编译,报错。\markdownRendererUlItemEnd 
\markdownRendererUlItem 根据教务处的新要求调整附录部分。\markdownRendererUlItemEnd 
\markdownRendererUlItem 参考文献中杂志文章如果没有卷号,那么页码直接跟在 年份后面,并用句点分割。在 \markdownRendererCodeSpan{thubib.bst} 中增加 output.year 函数。\markdownRendererUlItemEnd 
\markdownRendererUlItem 如果没有设置格式选项,报错。\markdownRendererUlItemEnd 
\markdownRendererUlItem submit 只能由本科用。\markdownRendererUlItemEnd 
\markdownRendererUlItem 研究生院目录要 times,而教务处要 arial。\markdownRendererUlItemEnd 
\markdownRendererUlItem 本科 openright,研究生 openany。\markdownRendererUlItemEnd 
\markdownRendererUlItem \markdownRendererCodeSpan{\markdownRendererBackslash{}backmatter}: 本科正文之后页码即用罗马数字,研究生不变。\markdownRendererUlItemEnd 
\markdownRendererUlItem \markdownRendererCodeSpan{\markdownRendererBackslash{}thu@textcircled}: 脚注编号使用 \markdownRendererCodeSpan{\markdownRendererBackslash{}textcircled} 命令,每页允许至多 99 个。\markdownRendererUlItemEnd 
\markdownRendererUlItem 本科公式编号前添加\markdownRendererStrongEmphasis{公式}二字。需要修 \markdownRendererCodeSpan{amsmath} 极其深的一个命令。\markdownRendererUlItemEnd 
\markdownRendererUlItem 教务处居然要本科论文公式全文编号!\markdownRendererUlItemEnd 
\markdownRendererUlItem 增加 \markdownRendererCodeSpan{subfigure} 和 \markdownRendererCodeSpan{subtable} 的 caption 配置。\markdownRendererUlItemEnd 
\markdownRendererUlItem 重新定义表格默认字体。\markdownRendererUlItemEnd 
\markdownRendererUlItem 让 \markdownRendererCodeSpan{\markdownRendererBackslash{}chapter*} 自动 \markdownRendererCodeSpan{\markdownRendererBackslash{}markboth}。\markdownRendererUlItemEnd 
\markdownRendererUlItem \markdownRendererCodeSpan{\markdownRendererBackslash{}tableofcontents}: 减小目录项中的导引小点跟页码之间的留白。\markdownRendererUlItemEnd 
\markdownRendererUlItem 硕士封面的冒号前居然有点小距离!\markdownRendererUlItemEnd 
\markdownRendererUlItem \markdownRendererCodeSpan{\markdownRendererBackslash{}thu@first@titlepage}: 本科封面标题调整微小的空隙。\markdownRendererUlItemEnd 
\markdownRendererUlItem \markdownRendererCodeSpan{\markdownRendererBackslash{}thu@first@titlepage}: 本科封面标题第二行的横线上移一点。\markdownRendererUlItemEnd 
\markdownRendererUlItem \markdownRendererCodeSpan{\markdownRendererBackslash{}thu@makeabstract}: 教务处又不要正文前的页眉了。\markdownRendererUlItemEnd 
\markdownRendererUlItem \markdownRendererCodeSpan{\markdownRendererBackslash{}thu@makeabstract}: 不管是哪种论文格式,摘要都要右开。\markdownRendererUlItemEnd 
\markdownRendererUlItem \markdownRendererCodeSpan{\markdownRendererBackslash{}thu@makeabstract}: 研究生封面英文摘要连续。\markdownRendererUlItemEnd 
\markdownRendererUlItem \markdownRendererCodeSpan{\markdownRendererBackslash{}listofequations*}: 公式索引项 numwidth 增加。\markdownRendererUlItemEnd 
\markdownRendererUlItem \markdownRendererCodeSpan{resume}: 教务处和研究生院非要搞的不一样!\markdownRendererUlItemEnd 
\markdownRendererUlEnd \markdownRendererInterblockSeparator
{}\markdownRendererHeadingTwo{\markdownRendererLink{v2.5}{https://github.com/tuna/thuthesis/compare/v2.4.2...v2.5}{https://github.com/tuna/thuthesis/compare/v2.4.2...v2.5}{} - 2006-05-20}\markdownRendererInterblockSeparator
{}\markdownRendererHeadingThree{Changed}\markdownRendererInterblockSeparator
{}\markdownRendererUlBegin
\markdownRendererUlItem 对本科论文进行大幅度的重写,因为教务处修改了格式要求。\markdownRendererUlItemEnd 
\markdownRendererUlItem 重新整理代码,使其布局更易读。\markdownRendererUlItemEnd 
\markdownRendererUlItem 增加本科论文的提交选项 submit。\markdownRendererUlItemEnd 
\markdownRendererUlItem \markdownRendererCodeSpan{\markdownRendererBackslash{}ps@thu@headings}: 本科的奇偶页眉不同。\markdownRendererUlItemEnd 
\markdownRendererUlItem \markdownRendererCodeSpan{\markdownRendererBackslash{}ps@thu@headings}: 增加 empty 页面样式。\markdownRendererUlItemEnd 
\markdownRendererUlItem 修正 minipage 中的脚注。\markdownRendererUlItemEnd 
\markdownRendererUlItem 标题上下间距重调,以前没有考虑 \markdownRendererCodeSpan{\markdownRendererBackslash{}intextsep} 的影响。\markdownRendererUlItemEnd 
\markdownRendererUlItem 增加索引名称定义。\markdownRendererUlItemEnd 
\markdownRendererUlItem 取消 \markdownRendererCodeSpan{titlesec} 宏包,用基本 LaTeX 命令格式化标题。\markdownRendererUlItemEnd 
\markdownRendererUlItem \markdownRendererCodeSpan{\markdownRendererBackslash{}tableofcontents}: 取消 \markdownRendererCodeSpan{titletoc} 宏包,用 \markdownRendererCodeSpan{\markdownRendererBackslash{}dottedtocline} 调整 目录。\markdownRendererUlItemEnd 
\markdownRendererUlItem 院系和专业分别改名用 department 和 major,代替原来 的 affil 和 subject。\markdownRendererUlItemEnd 
\markdownRendererUlItem \markdownRendererCodeSpan{\markdownRendererBackslash{}makecover}: 本科论文评语位置调整。\markdownRendererUlItemEnd 
\markdownRendererUlItem \markdownRendererCodeSpan{\markdownRendererBackslash{}makecover}: 综合论文训练在授权说明之后。\markdownRendererUlItemEnd 
\markdownRendererUlItem \markdownRendererCodeSpan{acknowledgement}: 本科论文要求致谢声明分页,但是研究生的不分。\markdownRendererUlItemEnd 
\markdownRendererUlItem \markdownRendererCodeSpan{\markdownRendererBackslash{}listoftables*}: 增加插图、表格和公式索引。\markdownRendererUlItemEnd 
\markdownRendererUlItem \markdownRendererCodeSpan{\markdownRendererBackslash{}listoftables*}: 为了让索引中能出现\markdownRendererStrongEmphasis{图 xxx},不得不修改 LaTeX内部命令 \markdownRendererCodeSpan{\markdownRendererBackslash{}@caption}。\markdownRendererUlItemEnd 
\markdownRendererUlItem \markdownRendererCodeSpan{\markdownRendererBackslash{}equcaption}: 将公式编号写入临时文件以便生成公式列表。\markdownRendererUlItemEnd 
\markdownRendererUlItem \markdownRendererCodeSpan{\markdownRendererBackslash{}listofequations*}: 增加公式索引命令。\markdownRendererUlItemEnd 
\markdownRendererUlItem 参考文献序号靠左,而不是靠右。\markdownRendererUlItemEnd 
\markdownRendererUlItem 不用 \markdownRendererCodeSpan{\markdownRendererBackslash{}CJKcaption},在导言区直接引入配置文件。\markdownRendererUlItemEnd 
\markdownRendererUlEnd \markdownRendererInterblockSeparator
{}\markdownRendererHeadingTwo{\markdownRendererLink{v2.4.2}{https://github.com/tuna/thuthesis/compare/v2.4.1...v2.4.2}{https://github.com/tuna/thuthesis/compare/v2.4.1...v2.4.2}{} - 2006-04-18}\markdownRendererInterblockSeparator
{}\markdownRendererHeadingThree{Changed}\markdownRendererInterblockSeparator
{}\markdownRendererUlBegin
\markdownRendererUlItem 去掉参考文献第二个作者后面烦人的逗号。\markdownRendererUlItemEnd 
\markdownRendererUlEnd \markdownRendererInterblockSeparator
{}\markdownRendererHeadingTwo{\markdownRendererLink{v2.4.1}{https://github.com/tuna/thuthesis/compare/v2.4...v2.4.1}{https://github.com/tuna/thuthesis/compare/v2.4...v2.4.1}{} - 2006-04-17}\markdownRendererInterblockSeparator
{}\markdownRendererHeadingThree{Changed}\markdownRendererInterblockSeparator
{}\markdownRendererUlBegin
\markdownRendererUlItem 2.4 忘了把关键词的 tabular 改成 thu@tabular。\markdownRendererUlItemEnd 
\markdownRendererUlItem 参考文献最后一个作者前是逗号而不是 and。\markdownRendererUlItemEnd 
\markdownRendererUlEnd \markdownRendererInterblockSeparator
{}\markdownRendererHeadingTwo{\markdownRendererLink{v2.4}{https://github.com/tuna/thuthesis/compare/v2.3...v2.4}{https://github.com/tuna/thuthesis/compare/v2.3...v2.4}{} - 2006-04-15}\markdownRendererInterblockSeparator
{}\markdownRendererHeadingThree{Changed}\markdownRendererInterblockSeparator
{}\markdownRendererUlBegin
\markdownRendererUlItem Fill more pdf info. with \markdownRendererCodeSpan{\markdownRendererBackslash{}hypersetup}.\markdownRendererUlItemEnd 
\markdownRendererUlItem 自动隐藏密级为内部时后面的五角星。\markdownRendererUlItemEnd 
\markdownRendererUlItem 增加“注释(Remark)”环境。\markdownRendererUlItemEnd 
\markdownRendererUlItem 压缩 item 之间的距离。\markdownRendererUlItemEnd 
\markdownRendererUlItem \markdownRendererCodeSpan{thubib.bst} 文献标题取消自动小写。\markdownRendererUlItemEnd 
\markdownRendererUlItem 中文参考文献取消 In: Proceedings。\markdownRendererUlItemEnd 
\markdownRendererUlItem 英文文参考文献调整 In: editor, Proceedings。\markdownRendererUlItemEnd 
\markdownRendererUlItem 参考文献为学位论文时,加方括号,作者后面为实心点。\markdownRendererUlItemEnd 
\markdownRendererUlItem 中文参考文献作者超过三个加等。\markdownRendererUlItemEnd 
\markdownRendererUlItem 中文参考文献需要在 bib 中指定 \markdownRendererCodeSpan{lang="chinese"}。\markdownRendererUlItemEnd 
\markdownRendererUlItem 学位论文不在需要 type 字段。\markdownRendererUlItemEnd 
\markdownRendererUlItem 为摘要等条目增加书签。\markdownRendererUlItemEnd 
\markdownRendererUlItem 章节的编号用黑体,也就是自动打开 \markdownRendererCodeSpan{arialtitle} 选项。\markdownRendererUlItemEnd 
\markdownRendererUlItem 添加模板名称命令。\markdownRendererUlItemEnd 
\markdownRendererUlItem 把页面尺寸写入 dvi,避免有的用户通 过 dvips 不指定页面类型而得到古怪的结果。\markdownRendererUlItemEnd 
\markdownRendererUlItem 表格内容为 11 磅。\markdownRendererUlItemEnd 
\markdownRendererUlItem 图表标题左对齐,取消原先漂亮的 hang 模式。\markdownRendererUlItemEnd 
\markdownRendererUlItem \markdownRendererCodeSpan{\markdownRendererBackslash{}thu@makeabstract}: It is \markdownRendererStrongEmphasis{Key words}, but not \markdownRendererStrongEmphasis{Key Words}.\markdownRendererUlItemEnd 
\markdownRendererUlItem \markdownRendererCodeSpan{denotation}: 为主要符号表环境增加一个可选参数,调节符号列的宽度。\markdownRendererUlItemEnd 
\markdownRendererUlItem \markdownRendererCodeSpan{acknowledgement}: 调整\markdownRendererStrongEmphasis{致谢}等中间的距离。\markdownRendererUlItemEnd 
\markdownRendererUlItem 参考文献间距调小一点,label 长度增加一点,以便让超过 100 的参考文献更好地对齐。\markdownRendererUlItemEnd 
\markdownRendererUlEnd \markdownRendererInterblockSeparator
{}\markdownRendererHeadingTwo{\markdownRendererLink{v2.3}{https://github.com/tuna/thuthesis/compare/v2.2...v2.3}{https://github.com/tuna/thuthesis/compare/v2.2...v2.3}{} - 2006-04-09}\markdownRendererInterblockSeparator
{}\markdownRendererHeadingThree{Changed}\markdownRendererInterblockSeparator
{}\markdownRendererUlBegin
\markdownRendererUlItem Fix a great bug: \markdownRendererCodeSpan{\markdownRendererBackslash{}PassOptionsToClass} and \markdownRendererCodeSpan{\markdownRendererBackslash{}LoadClass} rather than \markdownRendererCodeSpan{\markdownRendererBackslash{}PassOptionToPackage} and \markdownRendererCodeSpan{\markdownRendererBackslash{}LoadPackage}.\markdownRendererUlItemEnd 
\markdownRendererUlItem Reorganize the codes in cover, make the pagestyle more readable.\markdownRendererUlItemEnd 
\markdownRendererUlItem Add gbk2uni into the document.\markdownRendererUlItemEnd 
\markdownRendererUlItem Support \markdownRendererCodeSpan{openright} and openany.\markdownRendererUlItemEnd 
\markdownRendererUlItem Adjust \markdownRendererCodeSpan{\markdownRendererBackslash{}hypersetup} to remove color and box.\markdownRendererUlItemEnd 
\markdownRendererUlItem Adjust margins again.\markdownRendererUlItemEnd 
\markdownRendererUlItem Adjust references formats.\markdownRendererUlItemEnd 
\markdownRendererUlItem Redefine frontmatter and mainmatter to fit our case.\markdownRendererUlItemEnd 
\markdownRendererUlItem Add assumption environment.\markdownRendererUlItemEnd 
\markdownRendererUlItem Change the brace in the cover.\markdownRendererUlItemEnd 
\markdownRendererUlEnd \markdownRendererInterblockSeparator
{}\markdownRendererHeadingTwo{\markdownRendererLink{v2.2}{https://github.com/tuna/thuthesis/compare/v2.1...v2.2}{https://github.com/tuna/thuthesis/compare/v2.1...v2.2}{} - 2006-03-26}\markdownRendererInterblockSeparator
{}\markdownRendererHeadingThree{Changed}\markdownRendererInterblockSeparator
{}\markdownRendererUlBegin
\markdownRendererUlItem Adjust margins. How bad it is to simulate MS WORD!.\markdownRendererUlItemEnd 
\markdownRendererUlItem Add bachelor training overview details supporting.\markdownRendererUlItemEnd 
\markdownRendererUlItem CJK support in preamble.\markdownRendererUlItemEnd 
\markdownRendererUlItem Adjust hyperref to avoid boxes around links.\markdownRendererUlItemEnd 
\markdownRendererUlEnd \markdownRendererInterblockSeparator
{}\markdownRendererHeadingTwo{\markdownRendererLink{v2.1}{https://github.com/tuna/thuthesis/compare/v2.0e...v2.1}{https://github.com/tuna/thuthesis/compare/v2.0e...v2.1}{} - 2006-03-03}\markdownRendererInterblockSeparator
{}\markdownRendererHeadingThree{Changed}\markdownRendererInterblockSeparator
{}\markdownRendererUlBegin
\markdownRendererUlItem Add support to bachelor thesis.\markdownRendererUlItemEnd 
\markdownRendererUlItem Remove \markdownRendererCodeSpan{fancyhdr} and \markdownRendererCodeSpan{geometry}.\markdownRendererUlItemEnd 
\markdownRendererUlItem Redefine footnote marks.\markdownRendererUlItemEnd 
\markdownRendererUlItem Replace \markdownRendererCodeSpan{thubib.bst} with \markdownRendererCodeSpan{chinesebst.bst}.\markdownRendererUlItemEnd 
\markdownRendererUlItem Merge the modification of \markdownRendererCodeSpan{ntheorem}.\markdownRendererUlItemEnd 
\markdownRendererUlItem Remove \markdownRendererCodeSpan{footmisc} and refine the document.\markdownRendererUlItemEnd 
\markdownRendererUlItem Work very hard on the document.\markdownRendererUlItemEnd 
\markdownRendererUlItem Add \markdownRendererCodeSpan{\markdownRendererBackslash{}checklab} code to reduce “unresolved labels“ warning\markdownRendererUlItemEnd 
\markdownRendererUlItem \markdownRendererCodeSpan{\markdownRendererBackslash{}ps@thu@headings}: 彻底放弃 fancyhdr,定义自己的样式。\markdownRendererUlItemEnd 
\markdownRendererUlItem 让脚注它悬挂起来,而且中文中用上标,脚注中用正体。\markdownRendererUlItemEnd 
\markdownRendererUlItem \markdownRendererCodeSpan{\markdownRendererBackslash{}thu@first@titlepage}: 增加本科部分。\markdownRendererUlItemEnd 
\markdownRendererUlItem \markdownRendererCodeSpan{\markdownRendererBackslash{}makecover}: 分成几个小模块来搞,不然这个 macro 太大了,看不过来。\markdownRendererUlItemEnd 
\markdownRendererUlEnd \markdownRendererInterblockSeparator
{}\markdownRendererHeadingTwo{\markdownRendererLink{v2.0e}{https://github.com/tuna/thuthesis/compare/v2.0...v2.0e}{https://github.com/tuna/thuthesis/compare/v2.0...v2.0e}{} - 2005-12-18}\markdownRendererInterblockSeparator
{}\markdownRendererHeadingThree{Changed}\markdownRendererInterblockSeparator
{}\markdownRendererUlBegin
\markdownRendererUlItem \markdownRendererCodeSpan{denotation}: 主要符号表定义为一个 list,用起来方便。\markdownRendererUlItemEnd 
\markdownRendererUlEnd \markdownRendererInterblockSeparator
{}\markdownRendererHeadingTwo{\markdownRendererLink{v2.0}{https://github.com/tuna/thuthesis/compare/v1.5...v2.0}{https://github.com/tuna/thuthesis/compare/v1.5...v2.0}{} - 2005-12-20}\markdownRendererInterblockSeparator
{}\markdownRendererHeadingThree{Changed}\markdownRendererInterblockSeparator
{}\markdownRendererUlBegin
\markdownRendererUlItem \markdownRendererCodeSpan{\markdownRendererBackslash{}ps@thu@headings}: 以前的太乱了,重新整理过清晰多了。\markdownRendererUlItemEnd 
\markdownRendererUlItem \markdownRendererCodeSpan{\markdownRendererBackslash{}tableofcontents}: 附录的目录项需要调整一下。以及公式编号方式等等。\markdownRendererUlItemEnd 
\markdownRendererUlItem 增加了封面密级,增加博士封面支持\markdownRendererUlItemEnd 
\markdownRendererUlItem \markdownRendererCodeSpan{\markdownRendererBackslash{}thu@first@titlepage}: 封面的培养单位,学科等内容字距自动调整。\markdownRendererUlItemEnd 
\markdownRendererUlItem \markdownRendererCodeSpan{acknowledgement}: 将致谢定义为一个环境更合适,里面也不用像以前段首需 要自己缩进。\markdownRendererUlItemEnd 
\markdownRendererUlItem \markdownRendererCodeSpan{resume}: 最后决定将 resume 定义为环境。这样与前面的主要符号 表、致谢等对应。\markdownRendererUlItemEnd 
\markdownRendererUlEnd \markdownRendererInterblockSeparator
{}\markdownRendererHeadingTwo{\markdownRendererLink{v1.5}{https://github.com/tuna/thuthesis/compare/v1.4rc1...v1.5}{https://github.com/tuna/thuthesis/compare/v1.4rc1...v1.5}{} - 2005-12-16}\markdownRendererInterblockSeparator
{}\markdownRendererHeadingThree{Changed}\markdownRendererInterblockSeparator
{}\markdownRendererUlBegin
\markdownRendererUlItem \markdownRendererCodeSpan{acknowledgement}: 在那些不显示编号的章节前面先执行一次 \markdownRendererCodeSpan{\markdownRendererBackslash{}cleardoublepage},使新开章节的页码到达正确的状态。否则会因为 \markdownRendererCodeSpan{\markdownRendererBackslash{}addcontentsline} 在 chapter 之前而导致目录页码错误。\markdownRendererUlItemEnd 
\markdownRendererUlItem \markdownRendererCodeSpan{resume}: 增加个人简历章节的命令,去掉主文件中需要重新 定义 \markdownRendererCodeSpan{\markdownRendererBackslash{}cleardoublepage} 和自己写 \markdownRendererCodeSpan{\markdownRendererBackslash{}markboth},\markdownRendererCodeSpan{\markdownRendererBackslash{}addcontentsline} 的部分。\markdownRendererUlItemEnd 
\markdownRendererUlEnd \markdownRendererInterblockSeparator
{}\markdownRendererHeadingTwo{\markdownRendererLink{v1.4rc1}{https://github.com/tuna/thuthesis/compare/v1.4...v1.4rc1}{https://github.com/tuna/thuthesis/compare/v1.4...v1.4rc1}{} - 2005-12-14}\markdownRendererInterblockSeparator
{}\markdownRendererHeadingThree{Changed}\markdownRendererInterblockSeparator
{}\markdownRendererUlBegin
\markdownRendererUlItem I do not know why \markdownRendererCodeSpan{\markdownRendererBackslash{}thu@authorizationaddon} does not work now for v1.3, while it's fine in v1.2. Temporarily, I remove the directive :(. There might be better solution. Other changes: add \markdownRendererCodeSpan{config} option to subfig to be compatible with subfigure. add \markdownRendererCodeSpan{courier} package for tt font.\markdownRendererUlItemEnd 
\markdownRendererUlItem I have to put all chinese chars into cfg, otherwise they would not appear.\markdownRendererUlItemEnd 
\markdownRendererUlEnd \markdownRendererInterblockSeparator
{}\markdownRendererHeadingTwo{\markdownRendererLink{v1.4}{https://github.com/tuna/thuthesis/compare/v1.3...v1.4}{https://github.com/tuna/thuthesis/compare/v1.3...v1.4}{} - 2005-12-05}\markdownRendererInterblockSeparator
{}\markdownRendererHeadingThree{Changed}\markdownRendererInterblockSeparator
{}\markdownRendererUlBegin
\markdownRendererUlItem Fix the problem of \markdownRendererStrongEmphasis{chinese}, which is because both CJK and everysel redefine the \markdownRendererCodeSpan{\markdownRendererBackslash{}selectfont}. So, a not so good workaround is to merge them up. Add \markdownRendererCodeSpan{shuji.tex} example. Add \markdownRendererCodeSpan{\markdownRendererBackslash{}pozhehao} command.\markdownRendererUlItemEnd 
\markdownRendererUlEnd \markdownRendererInterblockSeparator
{}\markdownRendererHeadingTwo{\markdownRendererLink{v1.3}{https://github.com/tuna/thuthesis/compare/v1.2...v1.3}{https://github.com/tuna/thuthesis/compare/v1.2...v1.3}{} - 2005-11-14}\markdownRendererInterblockSeparator
{}\markdownRendererHeadingThree{Changed}\markdownRendererInterblockSeparator
{}\markdownRendererUlBegin
\markdownRendererUlItem Replace \markdownRendererCodeSpan{subfigure} with \markdownRendererCodeSpan{subfig}, replace \markdownRendererCodeSpan{caption2} with \markdownRendererCodeSpan{caption}, add details about using figure are in the example.\markdownRendererUlItemEnd 
\markdownRendererUlEnd \markdownRendererInterblockSeparator
{}\markdownRendererHeadingTwo{\markdownRendererLink{v1.2}{https://github.com/tuna/thuthesis/compare/v1.1...v1.2}{https://github.com/tuna/thuthesis/compare/v1.1...v1.2}{} - 2005-11-04}\markdownRendererInterblockSeparator
{}\markdownRendererHeadingThree{Changed}\markdownRendererInterblockSeparator
{}\markdownRendererUlBegin
\markdownRendererUlItem Remove \markdownRendererCodeSpan{fancyref}; Remove \markdownRendererCodeSpan{ucite} and implement \markdownRendererCodeSpan{\markdownRendererBackslash{}onlinecite}; use package \markdownRendererCodeSpan{arial} or \markdownRendererCodeSpan{helvet} selectively.\markdownRendererUlItemEnd 
\markdownRendererUlEnd \markdownRendererInterblockSeparator
{}\markdownRendererHeadingTwo{\markdownRendererLink{v1.1}{https://github.com/tuna/thuthesis/compare/v1.0...v1.1}{https://github.com/tuna/thuthesis/compare/v1.0...v1.1}{} - 2005-11-03}\markdownRendererInterblockSeparator
{}\markdownRendererHeadingThree{Changed}\markdownRendererInterblockSeparator
{}\markdownRendererUlBegin
\markdownRendererUlItem Initial version, migrate from the old ``Bao--Pan'' version. Make the template a class instead of package.\markdownRendererUlItemEnd 
\markdownRendererUlEnd \markdownRendererInterblockSeparator
{}\markdownRendererHeadingTwo{\markdownRendererLink{v1.0}{https://github.com/tuna/thuthesis/releases/tag/v1.0}{https://github.com/tuna/thuthesis/releases/tag/v1.0}{} - 2005-07-06}\markdownRendererInterblockSeparator
{}\markdownRendererHeadingThree{Changed}\markdownRendererInterblockSeparator
{}\markdownRendererUlBegin
\markdownRendererUlItem Please refer to ``Bao--Pan'' version.\markdownRendererUlItemEnd 
\markdownRendererUlEnd \relax